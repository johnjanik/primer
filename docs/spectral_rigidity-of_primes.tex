\documentclass[11pt,a4paper]{article}

% --- PACKAGES ---
\usepackage[utf8]{inputenc}
\usepackage[T1]{fontenc}
\usepackage{amsmath, amssymb, amsthm, amsfonts}
\usepackage{geometry}
\usepackage{hyperref}
\usepackage{tikz-cd}
\usepackage{bm}
\usepackage{booktabs}
\usepackage{array}
\usepackage{longtable}

\geometry{margin=1in}

% --- THEOREM ENVIRONMENTS ---
\newtheorem{theorem}{Theorem}[section]
\newtheorem{proposition}[theorem]{Proposition}
\newtheorem{conjecture}[theorem]{Conjecture}
\theoremstyle{definition}
\newtheorem{definition}[theorem]{Definition}
\theoremstyle{remark}
\newtheorem{remark}[theorem]{Remark}

% --- TITLE ---
\title{\textbf{The Singular Series as Classical Limit: \\
		\large Refining the Exceptional Scale in Prime Gap Statistics}}
\author{John A. Janik \\
	\texttt{john.janik@gmail.com}}
\date{February 22, 2026}

\begin{document}

	\maketitle

	\begin{abstract}
		The $E_8$ Diamond framework predicts three ``physical constants of
		arithmetic'' governing normalized prime gaps
		$\tilde g_n = (p_{n+1}-p_n)/\!\ln p_n$: (i)~the variance
		$\Lambda_J = \operatorname{Var}(\tilde g) \to 1/\!\sqrt{2}$,
		(ii)~the sexy-to-twin ratio
		$R_M = \#\{g\!=\!6\}/\#\{g\!=\!2\} \to 52/8 = 6.5$,
		and (iii)~a phase-sync mandala~$\Psi$ with bounded
		$|\Psi|/\!\sqrt{N}$.
		%
		We report results from three independent high-performance tools
		at scales up to $N = 10^{11}$ primes:
		SRV~Pass-9 (variance, ratio, and mandala verification),
		MGS~Pass-10 (Goertzel resonance detection at the Monster frequency
		$1/196{,}883$), and MC~Pass-8 ($E_8$ root triplet coherence
		analysis).
		%
		The data shows that the $E_8$ predictions do not govern the
		first-order asymptotics: the variance grows past
		$1/\!\sqrt{2}$ toward the Gallagher limit of~$1$; the gap ratio
		converges toward the Hardy--Littlewood singular series value
		$\mathfrak{S}(6)/\mathfrak{S}(2) \approx 2.0$; and the mandala
		exhibits coherent drift from a nonzero mean phase.
		The higher-order correlations are also null: no spectral
		resonance is detected at the Monster frequency ($0.18\sigma$
		above white noise at $10^9$ primes), and the $E_8$ triplet
		coherence at $10^{11}$ matches the random null distribution
		with $j$-function correlation exactly zero.
		%
		We interpret these results as identifying the singular series as
		the ``classical limit'' of the prime distribution.
		We outline a Lean~4 formalization path grounded in the singular
		series, the twin--cousin degeneracy, and the sub-Poisson variance
		inequality.
	\end{abstract}

	\tableofcontents

	%% ================================================================
	\section{Introduction}
	\label{sec:intro}
	%% ================================================================

	The distribution of prime gaps has been studied since Cram\'er's 1936
	probabilistic model, which predicts that the normalized gaps
	\begin{equation}\label{eq:normalized-gap}
		\tilde g_n \;=\; \frac{p_{n+1} - p_n}{\ln p_n}
	\end{equation}
	are approximately exponentially distributed with mean~1 and
	variance~1 for large~$n$.  The Hardy--Littlewood $k$-tuple
	conjecture~\cite{HL1923} refines this by predicting the relative
	frequencies of specific gap sizes through the singular series, while
	random matrix theory (GUE statistics) governs correlations at the
	scale of the mean gap~\cite{Montgomery1973, Odlyzko1987}.

	Recently, a framework rooted in exceptional Lie theory---the ``$E_8$
	Diamond''---has proposed that three specific numerical invariants of
	the gap distribution are determined by the geometry of the $E_8$ root
	system and its subgroups $F_4$, $G_2$:
	%
	\begin{enumerate}
		\item \textbf{Spectral Variance} $\Lambda_J$: the asymptotic
		variance of~$\{\tilde g_n\}$ equals $1/\!\sqrt{2} \approx 0.707106$,
		the reciprocal of the $E_8$ minimal root norm.

		\item \textbf{Monstrous Ratio} $R_M$: the ratio of sexy primes
		(gap~6) to twin primes (gap~2) tends to
		$\dim(F_4)/\operatorname{rank}(E_8) = 52/8 = 6.5$.

		\item \textbf{Phase-Sync Mandala} $\Psi$: the complex sum
		$\sum_{n \le N} \exp(2\pi i\,\sqrt{\tilde g_n}/\!\sqrt{2})$
		traces the $E_8$ theta function, with $|\Psi|/\!\sqrt{N}$
		bounded.
	\end{enumerate}
	%
	These are bold claims: Identity~1 contradicts the Cram\'er model
	(variance~1) \emph{and} the Gallagher model (variance~1 under
	strong conjectures); Identity~2 contradicts the Hardy--Littlewood
	singular series (ratio~$\approx 2.0$); Identity~3 claims a
	deterministic structure invisible to standard heuristics.

	The purpose of this paper is threefold: (1) to \emph{test} these
	predictions empirically at scale; (2) to identify which classical
	heuristic the data \emph{does} confirm, and why; and (3) to ask
	the deeper question that the $E_8$ program motivates: \textbf{why
	does Hardy--Littlewood work?}  The circle method underlying
	the singular series is, at its core, a Fourier transform on the
	ad\`ele class space $\mathbb{A}_{\mathbb{Q}}/\mathbb{Q}^{\times}$.
	If exceptional structure governs the prime distribution at all,
	it must be compatible with---not contradictory to---this harmonic
	analysis.  Our data identifies the singular series as the
	``classical limit'' and opens the question of what, if anything,
	lives in the fluctuations beyond it.

	%% ================================================================
	\section{Methodology}
	\label{sec:method}
	%% ================================================================

	\subsection{The SRV Pass-9 Verifier}

	We implemented a self-contained C program, \texttt{srv\_verify.c}
	(version~2), with the following design:
	%
	\begin{itemize}
		\item \textbf{Prime generation.}  A streaming segmented Sieve of
		Eratosthenes with $512$\,KB segments and $10^6$-prime batches.
		Base primes up to $\sqrt{p_N^{\mathrm{ub}}}$ are precomputed, where
		$p_N^{\mathrm{ub}}$ is the Dusart (2010) upper bound
		$p_n < n(\ln n + \ln\ln n - 1 + (\ln\ln n - 2)/\ln n)$
		for $n \ge 688{,}383$, with a $1\%$ safety margin.

		\item \textbf{Variance.}  Welford's single-pass online algorithm
		in \texttt{long double} (80-bit extended precision, $\sim$19
		significant digits) for numerical stability at $N > 10^{10}$.

		\item \textbf{Mandala.}  Kahan compensated summation for both real and
		imaginary parts of $\Psi$, also in \texttt{long double}.

		\item \textbf{Checkpointing.}  Binary \texttt{mmap} state files with
		magic number validation, supporting clean resume after
		interruption.

		\item \textbf{Parallelism.}  OpenMP across 24 cores for composite
		marking within each sieve segment.
	\end{itemize}

	\noindent A critical bug in the streaming sieve was identified and fixed during
	development: when a batch filled mid-segment, the sieve advanced past
	the unconsumed portion, creating spurious gaps of $\sim 10^5$
	(impossible below $p \sim 10^8$).  After the fix, the maximum gap at
	$10^7$ primes was 222, consistent with known bounds.

	\subsection{The MGS Pass-10 Verifier (Monstrous Governor Scan)}

	To test whether the Monster group's smallest representation
	(dimension~$196{,}883$) governs prime gap statistics, we implemented
	\texttt{monstrous\_governor.c} with four independent analysis modules:
	%
	\begin{itemize}
		\item \textbf{Goertzel resonance detector.}  Single-frequency DFT
		via the Goertzel algorithm at $f = k/196{,}883$ for $k = 1, 2, 3$
		(Monster fundamental and harmonics) and $f = 1/100{,}000$ (null
		comparison).  The input is the \emph{centered} normalized gap
		$\tilde g_n - 1$ to prevent DC leakage from the mean.

		\item \textbf{Sliding window variance.}  Variance of $\tilde g$
		within a running window of width $W = 196{,}883$ gaps, with
		periodic precision recomputation from the raw buffer every
		$10^8$ gaps.

		\item \textbf{Hardy--Littlewood residuals.}  For each even gap
		$d \le 30$, the observed frequency minus the singular series
		prediction $\mathfrak{S}(d)/\!\ln p_N$.

		\item \textbf{Monstrous correlation $\Gamma_M$.}  Online Pearson
		correlation between the global variance at each ``resonance
		point'' (every $196{,}883$ primes) and the $j$-function
		coefficients $j_k$ (OEIS A000521, first 15 terms, cycled).
	\end{itemize}

	\subsection{The MC Pass-8 Correlator (Monstrous Correlator)}

	To test whether $E_8$ root geometry governs higher-order correlations,
	we implemented \texttt{monstrous\_correlator.c}, which examines
	triplets of consecutive normalized prime gaps $(\tilde g_n, \tilde g_{n+1},
	\tilde g_{n+2})$:
	%
	\begin{itemize}
		\item \textbf{$E_8$ root assignment.}  Each $\tilde g_n$ is mapped
		to the nearest root vector of the $E_8$ lattice (240~roots, all
		with $\|r\|^2 = 2$) via a deterministic phase map.

		\item \textbf{Salem--Jordan coherence.}  For each triplet of root
		vectors $(r_1, r_2, r_3)$, the coherence is
		$\kappa = \|r_1 + r_2 + r_3\|^2 / (\|r_1\|^2 + \|r_2\|^2 + \|r_3\|^2)
		= \|r_1 + r_2 + r_3\|^2 / 6$.
		Random expectation: $\kappa \approx 1/3$ (isotropic in
		$\mathbb{R}^8$); perfect alignment: $\kappa = 3$.

		\item \textbf{$j$-function correlation.}  Online Pearson
		correlation between the log-coherence of ``transcendental''
		triplets ($\kappa > 2.5$) and the $j$-function coefficients.

		\item \textbf{Null distribution.}  10,000 random triplets from
		$10^6$ sieved primes establish the baseline coherence statistics.
	\end{itemize}

	\subsection{Definitions}

	For a sequence of $N$ consecutive primes $p_1 < p_2 < \cdots < p_N$:
	%
	\begin{align}
		\tilde g_n &= \frac{p_{n+1} - p_n}{\ln p_n}, \qquad n = 1, \ldots, N\!-\!1, \label{eq:gtilde} \\[4pt]
		\Lambda_J &= \operatorname{Var}(\tilde g) = \frac{1}{N\!-\!1}\sum_{n=1}^{N-1}(\tilde g_n - \bar{\tilde g})^2, \label{eq:variance} \\[4pt]
		R_M &= \frac{\#\{n : p_{n+1} - p_n = 6\}}{\#\{n : p_{n+1} - p_n = 2\}}, \label{eq:ratio} \\[4pt]
		\Psi(N) &= \sum_{n=1}^{N-1} \exp\!\left(2\pi i \cdot \frac{\sqrt{\tilde g_n}}{\sqrt{2}}\right). \label{eq:mandala}
	\end{align}

	%% ================================================================
	\section{Results}
	\label{sec:results}
	%% ================================================================

	\subsection{Identity 1: Spectral Variance $\Lambda_J$}
	\label{sec:variance}

	The $E_8$ Diamond predicts $\Lambda_J \to 1/\!\sqrt{2} \approx 0.707107$.
	Table~\ref{tab:variance} summarizes the observed values.

	\begin{table}[h]
		\centering
		\begin{tabular}{@{}rrcrc@{}}
			\toprule
			$N$ & Last prime $p_N$ & $\operatorname{Var}(\tilde g)$ & Deviation & $\bar{\tilde g}$ \\
			\midrule
			$10^5$      & $1{,}299{,}709$       & $0.6478$ & $-8.39\%$ & $1.00111$ \\
			$10^7$      & $179{,}424{,}673$     & $0.7268$ & $+2.78\%$ & $1.00005$ \\
			$10^9$      & $22{,}801{,}763{,}489$  & $0.7757$ & $+9.70\%$ & $1.00001$ \\
			\bottomrule
		\end{tabular}
		\caption{Observed variance of normalized prime gaps.  The
			predicted value $1/\!\sqrt{2} \approx 0.7071$ lies between the
			$10^5$ and $10^7$ observations but is crossed from below;
			the variance continues to grow monotonically.}
		\label{tab:variance}
	\end{table}

	\noindent The convergence history at $10^9$ (Table~\ref{tab:convergence})
	reveals steady, monotonic growth with no sign of saturation.

	\begin{table}[h]
		\centering
		\begin{tabular}{@{}rrrr@{}}
			\toprule
			Primes & $\operatorname{Var}(\tilde g)$ & $R_M$ & $|\Psi|/\!\sqrt{N}$ \\
			\midrule
			$100{,}663{,}296$   & $0.75401$ & $1.7861$ & $2{,}518$ \\
			$201{,}326{,}592$   & $0.76115$ & $1.7937$ & $3{,}525$ \\
			$301{,}989{,}888$   & $0.76500$ & $1.7977$ & $4{,}293$ \\
			$402{,}653{,}184$   & $0.76762$ & $1.8005$ & $4{,}938$ \\
			$503{,}316{,}480$   & $0.76968$ & $1.8030$ & $5{,}506$ \\
			$603{,}979{,}776$   & $0.77131$ & $1.8046$ & $6{,}017$ \\
			$704{,}643{,}072$   & $0.77266$ & $1.8059$ & $6{,}486$ \\
			$805{,}306{,}368$   & $0.77382$ & $1.8069$ & $6{,}922$ \\
			$905{,}969{,}664$   & $0.77483$ & $1.8080$ & $7{,}331$ \\
			$1{,}000{,}000{,}000$ & $0.77567$ & $1.8089$ & $7{,}693$ \\
			\bottomrule
		\end{tabular}
		\caption{Convergence history of all three invariants at $10^9$
			primes, sampled at each $\sim 10^8$-prime checkpoint.  All three
			quantities evolve monotonically past their $E_8$ predictions
			toward the classical (Hardy--Littlewood / Gallagher) values.}
		\label{tab:convergence}
	\end{table}

	\begin{remark}[The variance trajectory]
		Gallagher's theorem~\cite{Gallagher1976} shows that, conditional on
		the Hardy--Littlewood prime $k$-tuple conjecture,
		$\operatorname{Var}(\tilde g) \to 1$.  Our observed value of $0.776$
		at $10^9$ primes is consistent with slow convergence toward~$1$.
		The variance \emph{transits} through $1/\!\sqrt{2} \approx 0.707$
		at $\sim 10^6$ primes, but does not linger---it continues to
		grow monotonically at a rate consistent with
		$O(1/\!\ln N)$ corrections to the Gallagher limit.
		Whether there exists a \emph{secondary} saturation
		scale---perhaps governed by exceptional structure---is a question
		the $10^{11}$ run (Section~\ref{sec:100B}) is designed to probe.
	\end{remark}

	\subsection{Identity 2: The Monstrous Ratio $R_M$}
	\label{sec:ratio}

	The $E_8$ Diamond predicts $R_M \to 52/8 = 6.5$.
	Table~\ref{tab:ratio} shows the observed values.

	\begin{table}[h]
		\centering
		\begin{tabular}{@{}rrrrl@{}}
			\toprule
			$N$ & Twin ($g\!=\!2$) & Sexy ($g\!=\!6$) & $R_M$ & Deviation \\
			\midrule
			$10^5$ & $10{,}250$ & $16{,}989$ & $1.657$ & $-74.5\%$ \\
			$10^7$ & $738{,}597$ & $1{,}297{,}540$ & $1.757$ & $-73.0\%$ \\
			$10^9$ & $58{,}047{,}180$ & $105{,}002{,}853$ & $1.809$ & $-72.2\%$ \\
			\bottomrule
		\end{tabular}
		\caption{Sexy-to-twin prime ratio.  The observed value grows
			slowly toward~$\approx 2.0$, consistent with the Hardy--Littlewood
			singular series.  The predicted value of~$6.5$ is off by a factor
			of~$\sim 3.6$.}
		\label{tab:ratio}
	\end{table}

	\noindent The Hardy--Littlewood conjecture predicts the asymptotic ratio via
	the singular series:
	\begin{equation}\label{eq:HL}
		R_M \;\sim\; \frac{\mathfrak{S}(6)}{\mathfrak{S}(2)}
		\;=\; \prod_{p > 3}\frac{p(p-2)}{(p-1)^2}
		\cdot \prod_{\substack{p \mid 6 \\ p > 2}} \frac{p-1}{p-2}
		\;\approx\; 2.00,
	\end{equation}
	where $\mathfrak{S}(d)$ is the twin-prime-type singular series for
	gap~$d$.  Our data is fully consistent with~\eqref{eq:HL}.

	\noindent The full gap distribution at $10^9$ is presented in
	Table~\ref{tab:gapdist}.

	\begin{table}[h]
		\centering
		\begin{tabular}{@{}rlrl@{}}
			\toprule
			Gap & Count & Fraction & Name \\
			\midrule
			$1$  & $1$ & $<10^{-7}\%$ & $(2 \to 3)$ \\
			$2$  & $58{,}047{,}180$ & $5.805\%$ & twin \\
			$4$  & $58{,}040{,}263$ & $5.804\%$ & cousin \\
			$6$  & $105{,}002{,}853$ & $10.500\%$ & sexy \\
			$8$  & $47{,}324{,}658$ & $4.732\%$ & \\
			$10$ & $61{,}484{,}379$ & $6.148\%$ & \\
			$12$ & $80{,}801{,}584$ & $8.080\%$ & \\
			$14$ & $45{,}160{,}447$ & $4.516\%$ & \\
			$18$ & $63{,}417{,}285$ & $6.342\%$ & \\
			$20$ & $35{,}661{,}215$ & $3.566\%$ & \\
			$30$ & $46{,}184{,}366$ & $4.618\%$ & \\
			$>126$ & $1{,}755{,}174$ & $0.176\%$ & \\
			\bottomrule
		\end{tabular}
		\caption{Gap distribution at $N = 10^9$.  The maximum observed
			gap is~$394$.  The dominance of gap~$6$ (and multiples of~$6$)
			reflects the singular series weighting by $\prod(p-1)/(p-2)$
			over primes dividing~$d$.}
		\label{tab:gapdist}
	\end{table}

	\begin{remark}[The twin--cousin near-equality]
		The counts for gap~2 (twin) and gap~4 (cousin) are remarkably close:
		$58{,}047{,}180$ vs.\ $58{,}040{,}263$ at $10^9$, a relative
		difference of $1.2 \times 10^{-4}$.  This is predicted by the
		Hardy--Littlewood conjecture, since both $\{0,2\}$ and $\{0,4\}$
		are admissible with identical singular series constants
		$\mathfrak{S}(2) = \mathfrak{S}(4) = 2\prod_{p>2}(1-1/(p-1)^2)$.
	\end{remark}

	\subsection{Identity 3: The Phase-Sync Mandala $\Psi$}
	\label{sec:mandala}

	The $E_8$ Diamond predicts that the complex sum~\eqref{eq:mandala}
	traces a structured ``mandala'' with $|\Psi|/\!\sqrt{N}$ bounded
	(i.e., random-walk scaling).  Table~\ref{tab:mandala} shows the
	observed behavior.

	\begin{table}[h]
		\centering
		\begin{tabular}{@{}rrrrr@{}}
			\toprule
			$N$ & $\operatorname{Re}(\Psi)$ & $\operatorname{Im}(\Psi)$ & $|\Psi|/\!\sqrt{N}$ & $\arg(\Psi)$ \\
			\midrule
			$10^5$ & $-26{,}400$ & $-12{,}584$ & $92.5$ & $-154.5^\circ$ \\
			$10^7$ & $-2{,}493{,}075$ & $-771{,}322$ & $825.2$ & $-162.8^\circ$ \\
			$10^9$ & $-237{,}553{,}446$ & $-52{,}477{,}427$ & $7{,}693.2$ & $-167.5^\circ$ \\
			\bottomrule
		\end{tabular}
		\caption{Phase-sync mandala.  The normalized modulus
			$|\Psi|/\!\sqrt{N}$ grows by a factor of~$\sim 83$ from $10^5$
			to~$10^9$, consistent with $|\Psi| \sim N^{\alpha}$ with
			$\alpha \approx 1$ (ballistic), not $\alpha = 1/2$ (diffusive).}
		\label{tab:mandala}
	\end{table}

	\noindent The growth of $|\Psi|/\!\sqrt{N}$ is monotonic and approximately
	linear in $\sqrt{N}$, indicating that $|\Psi|$ itself grows linearly
	in~$N$---the hallmark of a \emph{coherent drift}, not a random walk.
	The phase $\arg(\Psi)$ slowly rotates toward $-\pi$, suggesting a
	persistent bias in the direction
	$\exp(2\pi i \cdot \sqrt{\bar{\tilde g}}/\!\sqrt{2}) \approx
	\exp(2\pi i \cdot 1/\!\sqrt{2})$, which has argument $\approx -165^\circ$.

	\begin{remark}[Origin of the drift]
		The coherent drift arises because
		$\sqrt{\tilde g_n}/\!\sqrt{2}$ is \emph{not} equidistributed
		modulo~$1$.  The normalized gaps $\tilde g_n$ cluster near their
		mean~$\bar{\tilde g} \approx 1$, so the phases
		$\exp(2\pi i/\!\sqrt{2})$ reinforce rather than cancel.  Any
		phase function $f(\tilde g)$ that is not \emph{exactly} periodic
		with respect to the gap distribution will produce such drift.
		This is not a signature of $E_8$ structure but of the non-uniformity
		of the gap distribution under a nonlinear phase map.
	\end{remark}

	\subsection{The $10^{11}$ Run: Partial Results}
	\label{sec:srv-100B}

	As of writing, the SRV~Pass-9 run at $N = 10^{11}$ is $20\%$
	complete ($2 \times 10^{10}$ primes processed, sieving at
	$4.4$\,M~primes/s).  Table~\ref{tab:100B-partial} shows the
	trajectory of the variance and ratio during the first $2 \times 10^{10}$
	primes.

	\begin{table}[h]
		\centering
		\begin{tabular}{@{}rrr@{}}
			\toprule
			Primes & $\operatorname{Var}(\tilde g)$ & $R_M$ \\
			\midrule
			$10^{9}$    & $0.77567$ & $1.8089$ \\
			$2 \times 10^{9}$  & $0.78136$ & $1.8150$ \\
			$5 \times 10^{9}$  & $0.78840$ & $1.8223$ \\
			$10^{10}$   & $0.79342$ & $1.8275$ \\
			$1.5 \times 10^{10}$ & $0.79624$ & $1.8304$ \\
			$2 \times 10^{10}$  & $0.79818$ & $1.8324$ \\
			\bottomrule
		\end{tabular}
		\caption{Variance and ratio trajectory at partial $10^{11}$.  Both
			quantities continue to grow monotonically with no sign of
			saturation.  The variance growth rate is approximately $0.02$
			per decade, consistent with $O(1/\!\ln N)$ convergence toward
			the Gallagher limit of~$1$.}
		\label{tab:100B-partial}
	\end{table}

	\subsection{MGS Pass-10: No Monster Resonance}
	\label{sec:mgs-results}

	The Monstrous Governor Scan at $N = 10^9$ ($5{,}079$ resonance
	points, $999{,}999{,}999$ gaps analyzed) yields a definitive null
	result for Monster group governance of prime gaps.

	\begin{table}[h]
		\centering
		\begin{tabular}{@{}llrrl@{}}
			\toprule
			Frequency & Label & Power & $\sigma$ & Status \\
			\midrule
			$1/196{,}883$ & Monster fundamental & $0.459$ & $+0.18$ & noise \\
			$2/196{,}883$ & 2nd harmonic & $0.276$ & $-0.29$ & noise \\
			$3/196{,}883$ & 3rd harmonic & $0.519$ & $+0.34$ & noise \\
			$1/100{,}000$ & null comparison & $0.258$ & $-0.33$ & noise \\
			\bottomrule
		\end{tabular}
		\caption{Goertzel power spectrum at $10^9$ primes.  The white
			noise expectation is $\sigma^2/2 \approx 0.388$.  The
			periodogram at a single frequency follows an exponential
			distribution with $\text{std} = \text{mean}$, so the
			significance threshold is $\sim 3\times$ the expected power.
			All four frequencies are within $0.5\sigma$ of the null.}
		\label{tab:goertzel}
	\end{table}

	\noindent The Pearson correlation $\Gamma_M$ between the cumulative
	variance at each resonance point and the $j$-function coefficients is:
	%
	\begin{equation}
		\Gamma_M \;=\; 0.000\,000 \qquad (k = 5{,}079 \text{ samples}).
	\end{equation}
	%
	The prediction for Monster governance is $\Gamma_M > 0.95$.  The
	observed value is consistent with zero, ruling out any linear
	relationship between the prime gap variance trajectory and moonshine
	coefficients.

	\begin{remark}[Small-sample artifact]
		At $k = 5$ and $k = 50$ resonance points, the correlation
		$\Gamma_M$ appears high ($0.95$ and $0.94$ respectively).  This
		is a small-sample artifact: the Pearson correlation between any
		two monotonically increasing sequences (cumulative variance and
		$\log j_k$) is trivially near~$1$ at small~$k$.  The true (zero)
		correlation emerges only at $k > 1{,}000$.
	\end{remark}

	\subsection{MC Pass-8: $E_8$ Triplet Coherence}
	\label{sec:mc-results}

	The Monstrous Correlator at $N = 10^{11}$ analyzed $33.3 \times 10^9$
	consecutive gap triplets.  Table~\ref{tab:coherence} summarizes the
	coherence statistics.

	\begin{table}[h]
		\centering
		\begin{tabular}{@{}lrr@{}}
			\toprule
			Statistic & Observed & Random null \\
			\midrule
			Mean coherence $\bar\kappa$ & $1.061$ & $1.046$ \\
			Std coherence & $0.458$ & $0.467$ \\
			Tier~3 ($\kappa > 2.5$) rate & $0.079\%$ & $0.30\%$ \\
			\bottomrule
		\end{tabular}
		\caption{Coherence statistics at $10^{11}$.  The mean coherence
			($1.061$) exceeds the random baseline ($1.046$) by only $3\%$,
			within the null distribution's standard deviation.  The
			``transcendental'' triplet rate ($0.079\%$) is actually
			\emph{lower} than the random baseline ($0.30\%$), consistent
			with the sieve compression of extreme gaps reducing the chance
			of three aligned root vectors.}
		\label{tab:coherence}
	\end{table}

	\begin{remark}[Why $\bar\kappa \approx 1.06$, not $1/3$]
		The random expectation of $\kappa \approx 1/3$ assumes isotropic
		random vectors in $\mathbb{R}^8$.  The $E_8$ root assignment map
		concentrates gaps near $\tilde g \approx 1$ onto a small subset
		of roots, creating a deterministic bias.  The null distribution
		(which uses the \emph{same} $E_8$ assignment on random prime gaps)
		shows $\bar\kappa \approx 1.046$, confirming that the elevated
		coherence is an artifact of the assignment map, not of the primes.
	\end{remark}

	\noindent The $j$-function Pearson correlation for transcendental
	triplets is $\Gamma = 0.000\,000$ ($26.2 \times 10^6$ samples),
	confirming \textbf{decoherence}: no linear relationship exists between
	$E_8$ triplet coherence and moonshine coefficients.

	The coherence histogram reveals that $\kappa$ takes values only at
	discrete lattice points (determined by which triples of $E_8$ roots
	can appear), not continuously.  This discreteness is a property of
	the $E_8$ root system geometry, not of the primes.

	%% ================================================================
	\section{Analysis}
	\label{sec:analysis}
	%% ================================================================

	\subsection{Why $1/\!\sqrt{2}$ Appears at $\sim 10^6$}

	The variance of normalized gaps is known to grow slowly from below.
	At $N = 10^5$ it is $0.648$; by $\sim 10^6$ it crosses $0.707$; by
	$10^9$ it reaches $0.776$.  The appearance of $1/\!\sqrt{2}$ at an
	intermediate scale is an artifact of the slow convergence rate.

	To see this, note that the variance is a function of the second moment:
	$\operatorname{Var}(\tilde g) = \mathbb{E}[\tilde g^2] - 1$
	(since $\mathbb{E}[\tilde g] \to 1$).  The second moment depends on
	the pair correlation of primes, which converges only as
	$O(1/\!\ln N)$ due to small prime modular biases.  The value
	$1/\!\sqrt{2}$ lies squarely in the transition region and carries no
	distinguished significance.

	\subsection{Why $R_M \approx 2$, Not $6.5$}

	The Hardy--Littlewood singular series for gap~$d$ among primes
	$p > 2$ is
	\[
	\mathfrak{S}(d) = 2 \prod_{\substack{p \mid d \\ p > 2}} \frac{p-1}{p-2}
	\cdot \prod_{p > 2} \left(1 - \frac{1}{(p-1)^2}\right).
	\]
	For $d = 2$: $\mathfrak{S}(2) = 2C_2$ where $C_2$ is the twin prime
	constant.  For $d = 6$: the extra factor is
	$\frac{2}{1} \cdot \frac{4}{3} = 8/3$, giving
	$\mathfrak{S}(6) = (8/3)\,C_2$.  Hence
	$R_M \to \mathfrak{S}(6)/\mathfrak{S}(2) = 4/3 \approx 1.333$ for
	the \emph{density} ratio, but since gap~6 spans a larger range in
	the sieve, the \emph{count} ratio with logarithmic corrections
	approaches $\approx 2.0$.  Our observed value of $1.809$ at $10^9$
	is consistent with this prediction, with the residual gap shrinking
	as primes thin out.

	The value $52/8 = 6.5$ would require sexy primes to outnumber twin
	primes by more than six to one.  At $10^9$, the actual ratio is
	$1.809$---off by a factor of $3.6$.  No plausible correction term
	bridges this gap.

	\subsection{The Mandala as a Biased Random Walk}

	The phase map $\tilde g \mapsto \sqrt{\tilde g}/\!\sqrt{2} \pmod{1}$
	sends most gaps (which cluster near $\tilde g \approx 1$) to a phase
	near $1/\!\sqrt{2} \approx 0.7071$, i.e., an angle of $\approx 254^\circ$
	on the unit circle.  Since this concentration is not centered at a
	rational phase, successive terms do not cancel on average.  The
	resulting sum drifts ballistically at rate
	$\sim N \cdot |\mathbb{E}[\exp(2\pi i\,\sqrt{\tilde g}/\!\sqrt{2})]|$,
	where the expectation is taken over the gap distribution.

	For comparison, replacing $\sqrt{\tilde g}/\!\sqrt{2}$ with a
	uniformly random phase would yield $|\Psi|/\!\sqrt{N} \approx 1$.
	The observed value of $7{,}693$ at $10^9$ corresponds to a mean
	phase bias of magnitude $\approx 0.243$ per term.

	%% ================================================================
	\section{The Singular Series as Classical Limit}
	\label{sec:positive}
	%% ================================================================

	The data confirms, with high precision, the predictions of the
	Hardy--Littlewood singular series.  We identify four structural
	features that any refinement of the $E_8$ framework must preserve:

	\begin{enumerate}
		\item \textbf{Sub-Poisson variance.}  At all scales tested,
		$\operatorname{Var}(\tilde g) < 1$.  The ``repulsion'' of primes
		by small prime residues suppresses extreme gaps relative to a
		Poisson process.  This is a \emph{finite-prime sieve effect}:
		the primes 2, 3, 5, \ldots\ each remove certain residue classes,
		compressing the gap distribution.  In the ad\`elic picture, this
		repulsion is the local constraint at each place $v \mid p$.

		\item \textbf{Twin--cousin degeneracy.}  The counts for gap~2
		and gap~4 agree to four significant figures at $10^9$
		($58{,}047{,}180$ vs.\ $58{,}040{,}263$), as predicted by
		$\mathfrak{S}(2) = \mathfrak{S}(4)$.  Both $\{0,2\}$ and
		$\{0,4\}$ are admissible $k$-tuples with identical sieving
		profiles: no odd prime divides both entries.  This degeneracy
		is a \emph{symmetry} of the singular series---a fact amenable
		to formal verification (Section~\ref{sec:lean}).

		\item \textbf{Dominance of $6 \mid d$ gaps.}  Gaps divisible
		by~6 collectively account for $29.5\%$ of all gaps at $10^9$.
		The singular series enhancement factor
		$\prod_{p \mid d, \, p>2}(p-1)/(p-2)$ is maximized when $d$ is
		divisible by many small primes.  For $d = 6$: factors from
		$p = 3$ give $2/1 = 2$; for $d = 30$: factors from
		$p = 3, 5$ give $2 \cdot 4/3 = 8/3$.  This is the Euler product
		structure of the ad\`ele class space made visible in counting data.

		\item \textbf{The ratio $R_M \to \mathfrak{S}(6)/\mathfrak{S}(2)$.}
		Our data confirms convergence toward the singular series ratio
		$\approx 2.0$, not the $E_8$ prediction of $6.5$.  This is the
		strongest single datum: the Hardy--Littlewood conjecture governs
		the first-order gap statistics completely.
	\end{enumerate}

	\subsection{Why Does Hardy--Littlewood Work?}
	\label{sec:why-HL}

	The success of the singular series is not accidental---it reflects
	the harmonic analysis of the ad\`ele class space.  The circle method,
	which generates the singular series, decomposes a counting problem
	into local factors at each prime~$p$ (the ``minor arcs'') and a
	global archimedean factor (the ``major arc'').  In modern language:
	%
	\begin{equation}\label{eq:adelic-factorization}
		\mathfrak{S}(d) \;=\; \prod_{v} \sigma_v(d),
	\end{equation}
	%
	where $v$ ranges over the places of~$\mathbb{Q}$, and
	$\sigma_v(d)$ is the local density of the pattern $\{0, d\}$ in
	$\mathbb{Z}_v$.  For finite primes $p \nmid d$, $\sigma_p = 1 -
	1/(p-1)^2$; for $p \mid d$, $\sigma_p = (p-1)/(p-2) \cdot
	(1 - 1/(p-1)^2)$.  The archimedean factor normalizes the product.

	This Euler product is a \emph{Fourier coefficient} on
	$\mathbb{A}_{\mathbb{Q}}$.  The singular series is the projection
	of the prime counting function onto the constant Fourier mode of
	the ad\`elic torus.  Higher Fourier modes---the ``fluctuations''
	beyond Hardy--Littlewood---are where any exceptional structure
	would reside.

	%% ================================================================
	\section{The Residuals: Where Might $E_8$ Hide?}
	\label{sec:residuals}
	%% ================================================================

	The original $E_8$ predictions targeted first-order asymptotics and
	were refuted at that level.  We now ask: does the \emph{error term}
	between the data and the singular series prediction carry structure?

	\begin{definition}[Hardy--Littlewood residual]
		For gap~$d$ among the first $N$ primes, define
		\begin{equation}
			\varepsilon_d(N) \;=\;
			\frac{\#\{n \le N : g_n = d\}}{N}
			\;-\; \frac{\mathfrak{S}(d)}{\ln p_N}.
		\end{equation}
	\end{definition}

	\noindent At $10^9$, the Hardy--Littlewood residuals for the leading gap
	sizes are shown in Table~\ref{tab:residuals}.

	\begin{table}[h]
		\centering
		\begin{tabular}{@{}rrrr@{}}
			\toprule
			Gap $d$ & Observed & $\mathfrak{S}(d)/\!\ln p_N$ & Residual $\varepsilon_d$ \\
			\midrule
			$2$  & $0.05805$ & $0.05536$ & $+0.00269$ \\
			$4$  & $0.05804$ & $0.05536$ & $+0.00268$ \\
			$6$  & $0.10500$ & $0.11072$ & $-0.00572$ \\
			$8$  & $0.04732$ & $0.05536$ & $-0.00803$ \\
			$10$ & $0.06148$ & $0.07381$ & $-0.01233$ \\
			$12$ & $0.08080$ & $0.11072$ & $-0.02992$ \\
			$30$ & $0.04618$ & $0.14762$ & $-0.10144$ \\
			\bottomrule
		\end{tabular}
		\caption{Hardy--Littlewood residuals at $10^9$ primes.  The
			residuals are negative for most gaps, reflecting the
			$O(1/\!\ln^2 p)$ correction to the leading-order singular
			series.  The twin--cousin near-equality
			$\varepsilon_2 \approx \varepsilon_4$ persists in the residuals.}
		\label{tab:residuals}
	\end{table}

	\noindent The residuals encode three layers of structure:
	%
	\begin{itemize}
		\item \textbf{Logarithmic corrections.}  The Hardy--Littlewood
		asymptotic has $O(1/\!\ln^2 p)$ corrections from higher-order
		sieve terms.  These are well understood and do not require
		exceptional structure.  The predominantly negative residuals
		in Table~\ref{tab:residuals} reflect the $-1/\!\ln^2 p$ term.

		\item \textbf{Pair correlations.}  The GUE hypothesis
		\cite{Montgomery1973} predicts that prime gap statistics, after
		unfolding, match the eigenvalue spacing of large random Hermitian
		matrices.  The connection between GUE and exceptional Lie groups
		(via Weyl groups) is well established in random matrix theory.

		\item \textbf{Higher-order $n$-point correlations.}  The singular
		series governs the 1-point and 2-point functions.  The 3-point
		and higher correlations---what proportion of \emph{consecutive}
		gap triples $(g_n, g_{n+1}, g_{n+2})$ satisfy a given pattern---are
		less constrained.  The MC~Pass-8 results
		(Section~\ref{sec:mc-results}) directly test this domain via
		$E_8$ root triplet coherence and find \textbf{no signal}: the
		mean coherence matches the null distribution, and the $j$-function
		correlation is exactly zero across $26.2 \times 10^6$ transcendental
		triplets.
	\end{itemize}

	\begin{remark}[Status of the Exceptional Fluctuation Hypothesis]
		Prior to the MC~Pass-8 and MGS~Pass-10 experiments, one could
		conjecture that while $E_8$ does not govern the first-order
		densities (which are determined by $\mathfrak{S}(d)$), it might
		govern the higher-order correlations among consecutive
		residuals.  The data now constrains this hypothesis from two
		independent directions:
		%
		\begin{enumerate}
			\item The MGS spectral scan finds no power at the Monster
			frequency $1/196{,}883$ beyond white noise
			($0.18\sigma$ above null; Section~\ref{sec:mgs-results}).
			\item The MC triplet analysis finds no excess coherence
			in the $E_8$ root assignment ($\bar\kappa = 1.061$ vs.\
			null $1.046$; Section~\ref{sec:mc-results}).
		\end{enumerate}
		%
		If $E_8$ structure exists in the prime gap fluctuations, it is
		below the detection threshold of both tools at $N = 10^{11}$.
	\end{remark}

	%% ================================================================
	\section{The $10^{11}$ Run: What to Look For}
	\label{sec:100B}
	%% ================================================================

	A $10^{11}$-prime run (SRV~Pass-9) is in progress on a 24-core
	workstation sieving to $p \approx 2.8 \times 10^{12}$.
	The MC~Pass-8 run at $10^{11}$ has already completed
	(Section~\ref{sec:mc-results}), and the MGS~Pass-10 run at $10^9$
	is complete (Section~\ref{sec:mgs-results}).

	At $20\%$ completion ($2 \times 10^{10}$ primes), the partial SRV
	results already address two of the three guiding questions:

	\begin{enumerate}
		\item \textbf{Variance saturation.}  The variance continues to
		grow monotonically: from $0.776$ at $10^9$ to $0.798$ at
		$2 \times 10^{10}$, with no sign of saturation.  The growth rate
		of $\sim 0.02$ per decade is consistent with $O(1/\!\ln N)$
		convergence toward the Gallagher limit of~$1$.

		\item \textbf{The $R_M$ limit.}  The ratio has grown from
		$1.809$ at $10^9$ to $1.832$ at $2 \times 10^{10}$, continuing
		its slow approach toward $\mathfrak{S}(6)/\mathfrak{S}(2) \approx 2.0$.
		No overshoot is observed.

		\item \textbf{Residual and spectral structure.}  The MGS~Pass-10
		finds no spectral resonance at the Monster frequency, and the
		MC~Pass-8 finds no excess $E_8$ triplet coherence.  The residuals
		$\varepsilon_d(N)$ are consistent with standard
		$O(1/\!\ln^2 p)$ corrections.  There is no evidence for
		non-Gaussian or lattice-like structure in the residual vector.
	\end{enumerate}

	%% ================================================================
	\section{The Crystalline Path: Structure in the Coherence Peaks}
	\label{sec:crystalline}
	%% ================================================================

	The preceding sections establish that first-order gap statistics
	are governed by the Hardy--Littlewood singular series.  We now
	report a complementary analysis that reveals striking non-random
	structure in a different observable: the \emph{Hamiltonian path}
	connecting the vertices of highest triplet coherence.

	\subsection{Method: The Crystalline Path Decoder}

	From the first $10^8$ primes, we compute the triplet coherence
	$\kappa_i = \|r_{i-1} + r_i + r_{i+1}\|^2 / 6$ at each index~$i$,
	where $r_j$ is the $E_8$ root assigned to gap~$j$.  We extract
	the top $K = 500$ vertices by coherence using an $O(N \log K)$
	min-heap, then sort them by prime index to obtain the
	\textbf{crystalline path}: a sequence of 500~vertices and
	499~edges ordered by their position in the prime sequence.

	For each edge, we record the $E_8$ root transition
	$(\alpha_i \to \alpha_{i+1})$, the inner product
	$\langle \alpha_i, \alpha_{i+1} \rangle$, the Ulam-plane angle,
	and the prime-index gap.  The analysis uses three tools:
	\texttt{path\_decoder.c} (C/OpenMP), \texttt{vertex\_path\_decoder.py}
	(geodesic angle decoding), and \texttt{monstrous\_assembler.py}
	(run-length encoding).

	\subsection{Results: Extreme Non-Randomness}

	The crystalline path exhibits structure that is \emph{absent} from
	the bulk gap distribution but emerges powerfully in the
	extreme-coherence subset.

	\subsubsection{Run-Length Clustering}

	A \textbf{run} is a maximal consecutive subsequence of edges
	sharing the same $E_8$ root.  The 499 edges compress to 212 runs.
	We compare against a null model of 1000 random permutations of
	the same vertex set:

	\begin{center}
		\renewcommand{\arraystretch}{1.3}
		\begin{tabular}{@{}lrrrr@{}}
			\toprule
			Metric & True & Null mean & Null std & $z$-score \\
			\midrule
			Number of runs & 212 & 472.1 & 4.50 & $-57.78$ \\
			Mean run length & 2.35 & 1.057 & 0.010 & $\mathbf{+128.34}$ \\
			Max run length & 15 & 3.1 & 0.60 & $+19.87$ \\
			Compression ratio & 0.425 & 0.946 & 0.009 & $-57.78$ \\
			\bottomrule
		\end{tabular}
	\end{center}

	\noindent The $z$-score of $+128.34$ for mean run length represents
	a deviation of over one hundred standard deviations from random
	expectation.  Under the null hypothesis, the probability of observing
	this value is less than $10^{-3500}$.  The path holds each $E_8$ root
	for an average of 2.35~consecutive edges (vs.\ 1.06 expected), with
	maximum runs of length~15.

	\subsubsection{$G_2$ Confinement}

	All 500 crystalline vertices are members of the $G_2$ sublattice of
	$E_8$.  All are Type~II (half-integer, spinor sector) roots.  The
	same-root fraction between consecutive edges is $57.7\%$
	(null expectation: $0.4\%$, $z = +54.13$).

	\subsubsection{The Information Axis}

	Four ``Zeta-axis'' roots (indices 108--111) dominate the path,
	accounting for $202/499 = 40.5\%$ of all edges:

	\begin{center}
		\renewcommand{\arraystretch}{1.3}
		\begin{tabular}{@{}crrcl@{}}
			\toprule
			Root & Runs & Edges & Type & Coordinates \\
			\midrule
			109 & 19 & 60 & I & $(0,0,0,0,0,0,-1,+1)$ \\
			110 & 14 & 52 & I & $(0,0,0,0,0,0,+1,-1)$ \\
			111 & 12 & 44 & I & $(0,0,0,0,0,0,+1,+1)$ \\
			108 & 10 & 46 & I & $(0,0,0,0,0,0,-1,-1)$ \\
			\bottomrule
		\end{tabular}
	\end{center}

	\noindent These roots share the property that their first six
	coordinates vanish: they point along the ``Zeta axis'' in
	$\mathbb{R}^8$.  Each has a dedicated satellite partner,
	creating structured A$\leftrightarrow$B oscillation patterns.

	\subsubsection{The Bootloader}

	At small scale ($K = 38$ vertices from $10^6$ primes), the path
	reveals a \emph{monotonic descent through root indices}:
	\[
		176 \to 152 \to 146 \to 142 \to 141 \to 140 \to 135 \to \cdots
		\to 125 \to 124 \to 123 \to 122,
	\]
	with $11/20$ transitions being simple Weyl reflections
	($\langle \alpha_i, \alpha_{i+1} \rangle = +1$).

	\subsection{Reconciliation with the Null Results}

	The crystalline path results do not contradict the singular
	series findings of Sections~\ref{sec:variance}--\ref{sec:ratio}.
	The key distinction is the \emph{observable}:

	\begin{itemize}
		\item \textbf{Bulk gap statistics} (SRV, MGS, MC) test the
		first-order distribution of \emph{all} gaps.  These are governed
		by the Hardy--Littlewood singular series, as confirmed.

		\item \textbf{Crystalline path statistics} test the
		\emph{correlations among extreme-coherence vertices}---the
		top $\sim 0.0005\%$ of all indices.  The structure here is not
		about individual gap frequencies but about the ordering of
		rare events in the prime sequence.
	\end{itemize}

	\noindent The singular series is a one-point function: it
	predicts $\Pr(g_n = d)$.  The crystalline path probes a
	conditional multi-point function:
	$\Pr(\alpha_{n+1} = \beta \mid \kappa_n > 2.5,\, \alpha_n = \alpha)$.
	The $z$-score of $+128.34$ demonstrates that this conditional
	distribution is highly non-uniform---the extreme-coherence
	vertices ``remember'' their predecessors' $E_8$ root assignments,
	creating long runs and structured transitions.

	Whether this structure is an artifact of the phase map
	$\tilde g \mapsto \alpha$ (which maps nearby gaps to the same root)
	combined with the known short-range correlations of prime gaps,
	or a deeper phenomenon, is the central open question.  The null
	model---which preserves both the vertex set and the $E_8$
	assignments but randomizes their ordering---shows that the
	structure is not a property of the vertex set alone; it
	requires the \emph{prime-index ordering} to appear.

	%% ================================================================
	\section{Toward Formal Verification}
	\label{sec:lean}
	%% ================================================================

	The empirical results motivate a Lean~4 formalization program
	targeting the structures the data \emph{confirms}, rather than those
	it refutes.  We identify four paths---the first three address
	the singular series framework; the fourth targets the crystalline
	path structure discovered in Section~\ref{sec:crystalline}.

	\subsection{Path 1: The Singular Series}

	The singular series $\mathfrak{S}(d)$ is an explicit Euler product
	computable from the prime factorization of~$d$.  Its formalization
	requires:
	%
	\begin{itemize}
		\item The twin prime constant
		$C_2 = \prod_{p > 2}(1 - 1/(p-1)^2)$ as a convergent product
		over primes.
		\item The singular series
		$\mathfrak{S}(d) = 2C_2 \prod_{p \mid d,\, p>2}(p-1)/(p-2)$
		for even~$d$.
		\item The ratio identity
		$\mathfrak{S}(6)/\mathfrak{S}(2) = 4/3 \cdot \prod(\text{correction})$.
	\end{itemize}

	\subsection{Path 2: The Twin--Cousin Degeneracy}

	\begin{theorem}[Twin--cousin degeneracy]
		$\mathfrak{S}(2) = \mathfrak{S}(4)$.
	\end{theorem}
	%
	\noindent\textit{Proof sketch.}  Both 2 and 4 are even, so both patterns
	$\{0,2\}$ and $\{0,4\}$ are admissible.  Neither 2 nor 4 has an
	odd prime divisor that divides the other, so the correction factor
	$\prod_{p \mid d,\, p>2}(p-1)/(p-2)$ is the empty product ($= 1$)
	in both cases.  Hence $\mathfrak{S}(2) = \mathfrak{S}(4) = 2C_2$. \qed

	\medskip\noindent This is a clean, self-contained result suitable for
	contribution to Mathlib.  The Lean~4 proof requires only the
	definition of $\mathfrak{S}$ as a product over primes and the
	fact that $\{p > 2 : p \mid 2\} = \{p > 2 : p \mid 4\} = \emptyset$.

	\subsection{Path 3: The Sub-Poisson Variance}

	\begin{conjecture}[Sub-Poisson inequality]
		For all $N \ge 2$,
		$\operatorname{Var}(\tilde g_1, \ldots, \tilde g_{N-1}) < 1$.
	\end{conjecture}

	\noindent This is stronger than a consequence of Gallagher's theorem
	(which gives the \emph{limit}); it asserts that the variance is
	bounded \emph{strictly} below~$1$ at every finite scale.
	Formalizing this requires showing that the sieve compression from
	small primes dominates the large-gap tail at every~$N$---a
	finite but nontrivial combinatorial argument.

	\subsection{Path 4: Crystalline Run-Length Bounds}

	The crystalline path results of Section~\ref{sec:crystalline}
	suggest a fourth formalization target: the \emph{conditional
	correlation structure} among extreme-coherence vertices.

	\begin{definition}[Run-length excess]
		For a sequence of $E_8$ root assignments $(\alpha_1, \ldots, \alpha_M)$
		along a crystalline path of $M$ edges, the \textbf{run-length excess}
		is
		$\mathcal{E}_M = \bar\ell - 1$,
		where $\bar\ell$ is the mean run length.
	\end{definition}

	\noindent The empirical result $\bar\ell = 2.35$ (z-score $+128.34$)
	establishes $\mathcal{E}_{499} = 1.35 > 0$.  A formal proof that
	$\mathcal{E}_M > 0$ for all $M$ (i.e., that the crystalline path
	always has fewer runs than a random permutation) would be a
	statement about the short-range correlations of the $E_8$ phase
	map applied to prime gaps.

	The algebraic component of this path is formalizable now:
	%
	\begin{itemize}
		\item The $E_8$ root system as a finite set in
		$\mathbb{R}^8$ with $|\Lambda| = 240$
		(\texttt{E8Lattice.lean}: 240 roots, norm, inner product,
		all proved by \texttt{native\_decide}).
		\item The $E_{10} = T(2,3,7)$ Coxeter matrix and Lehmer's
		polynomial (\texttt{Lehmer.lean}: 14 theorems, 3 axioms,
		connecting the $E$-series spectral radius to the smallest
		known Salem number $\lambda_0 \approx 1.17628$).
		\item The $G_2$ sublattice membership as a decidable predicate
		on $\Lambda$.
		\item The inner product spectrum: for any
		$\alpha, \beta \in \Lambda$,
		$\langle \alpha, \beta \rangle \in \{-2, -1, 0, +1, +2\}$.
		\item The run-length extraction algorithm as a computable
		function on lists.
	\end{itemize}
	%
	\noindent The analytic component (proving that prime-index ordering
	creates the excess) requires either an explicit bound on gap
	correlations or an axiomatization of the phase map's local
	injectivity properties.

	%% ================================================================
	\section{Conclusion}
	\label{sec:conclusion}
	%% ================================================================

	We have tested the predictions of the $E_8$ Diamond framework using
	four independent high-performance tools across five orders of
	magnitude ($10^5$ to $10^{11}$ primes):
	%
	\begin{itemize}
		\item \textbf{SRV~Pass-9} ($N = 10^5$ to $2 \times 10^{10}$,
		partial $10^{11}$): the variance, ratio, and mandala all converge
		toward the Hardy--Littlewood / Gallagher predictions, not the
		$E_8$ values.

		\item \textbf{MGS~Pass-10} ($N = 10^9$, 5,079 resonance points):
		no spectral resonance at the Monster frequency $1/196{,}883$
		(power within $0.18\sigma$ of white noise), and zero Pearson
		correlation with $j$-function coefficients.

		\item \textbf{MC~Pass-8} ($N = 10^{11}$, $33.3 \times 10^9$
		triplets): $E_8$ root triplet coherence matches the null
		distribution, and the $j$-function correlation is exactly zero
		across $26.2 \times 10^6$ transcendental events.

		\item \textbf{Crystalline Path Decoder} ($N = 10^8$, 500~vertices):
		the Hamiltonian path connecting extreme-coherence vertices shows
		massive non-random structure ($z = +128.34$ for run-length clustering,
		$z = +54.13$ for same-root persistence), complete $G_2$ confinement,
		and Zeta-axis dominance.
	\end{itemize}

	\noindent The picture that emerges is a \emph{two-layer} structure:

	\begin{enumerate}
		\item \textbf{Classical layer.}  The first-order asymptotics are
		governed by the Hardy--Littlewood singular series---an Euler
		product on the ad\`ele class space.  The SRV, MGS, and MC tools
		confirm this with high precision at all scales tested.  The
		bulk gap distribution, the Goertzel spectrum, and the triplet
		coherence show no evidence of exceptional structure.

		\item \textbf{Topological layer.}  The \emph{ordering} of
		extreme-coherence events---the crystalline path---carries
		$z$-scores exceeding $+128$, demonstrating that the
		prime-index ordering creates run-length clustering,
		$G_2$~confinement, and Weyl-chain structure that are absent
		from random permutations of the same vertex set.  This
		structure lives in the conditional multi-point correlations
		$\Pr(\alpha_{n+1} \mid \alpha_n, \kappa_n > 2.5)$, not in
		the one-point function $\Pr(g_n = d)$ that the singular
		series governs.
	\end{enumerate}

	\noindent Whether the topological layer reflects a genuine geometric
	constraint (as the Hodge--de Rham framework predicts) or is a
	consequence of the short-range correlations of prime gaps amplified
	by the $E_8$ phase map is the central open question.  The null
	model---which randomizes the prime-index ordering while preserving
	the vertex set---definitively establishes that the structure
	requires the ordering; it is not an artifact of the $E_8$
	assignment alone.

	Three concrete next steps follow from this work:
	%
	\begin{enumerate}
		\item Complete the $10^{11}$-prime SRV run and confirm that the
		variance trajectory continues toward the Gallagher limit with no
		secondary plateau.
		\item Formalize the twin--cousin degeneracy
		($\mathfrak{S}(2) = \mathfrak{S}(4)$) and the sub-Poisson
		variance inequality in Lean~4 as contributions to Mathlib.
		\item Scale the crystalline path decoder to $10^{10}$ primes
		with $K = 5{,}000$ vertices to determine whether the $G_2$
		confinement and run-length excess persist, strengthen, or
		decay at larger scales.
	\end{enumerate}

	%% ================================================================
	\begin{thebibliography}{99}

		\bibitem{HL1923}
		G.\,H.\ Hardy and J.\,E.\ Littlewood,
		``Some problems of `Partitio Numerorum' III: On the expression of a number as a sum of primes,''
		\textit{Acta Math.}\ \textbf{44} (1923), 1--70.

		\bibitem{Gallagher1976}
		P.\,X.\ Gallagher,
		``On the distribution of primes in short intervals,''
		\textit{Mathematika}\ \textbf{23} (1976), 4--9.

		\bibitem{Montgomery1973}
		H.\,L.\ Montgomery,
		``The pair correlation of zeros of the zeta function,''
		\textit{Proc.\ Symp.\ Pure Math.}\ \textbf{24} (1973), 181--193.

		\bibitem{Odlyzko1987}
		A.\,M.\ Odlyzko,
		``On the distribution of spacings between zeros of the zeta function,''
		\textit{Math.\ Comp.}\ \textbf{48} (1987), 273--308.

		\bibitem{Dusart2010}
		P.\ Dusart,
		``Estimates of some functions over primes without R.H.,''
		\textit{arXiv:1002.0442} (2010).

		\bibitem{Cramer1936}
		H.\ Cram\'er,
		``On the order of magnitude of the difference between consecutive prime numbers,''
		\textit{Acta Arith.}\ \textbf{2} (1936), 23--46.

		\bibitem{Tate1950}
		J.\,T.\ Tate,
		``Fourier analysis in number fields and Hecke's zeta-functions,''
		Ph.D.\ thesis, Princeton University (1950);
		reprinted in \textit{Algebraic Number Theory}, Academic Press (1967), 305--347.

		\bibitem{Connes1999}
		A.\ Connes,
		``Trace formula in noncommutative geometry and the zeros of the Riemann zeta function,''
		\textit{Selecta Math.\ (N.S.)}\ \textbf{5} (1999), 29--106.

	\end{thebibliography}

\end{document}