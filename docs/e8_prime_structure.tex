\documentclass[12pt,a4paper]{article}

% ================================================================
% Packages
% ================================================================
\usepackage[utf8]{inputenc}
\usepackage[T1]{fontenc}
\usepackage{amsmath,amssymb,amsthm}
\usepackage{mathtools}
\usepackage{geometry}
\geometry{margin=1in}
\usepackage{hyperref}
\usepackage{cleveref}
\usepackage{algorithm}
\usepackage{algpseudocode}
\usepackage{booktabs}
\usepackage{graphicx}
\usepackage{xcolor}
\usepackage{enumitem}
\usepackage{microtype}
\usepackage{listings}
\usepackage{caption}
\usepackage{longtable}

% ================================================================
% Theorem environments
% ================================================================
\newtheorem{theorem}{Theorem}[section]
\newtheorem{proposition}[theorem]{Proposition}
\newtheorem{lemma}[theorem]{Lemma}
\newtheorem{corollary}[theorem]{Corollary}
\theoremstyle{definition}
\newtheorem{definition}[theorem]{Definition}
\newtheorem{example}[theorem]{Example}
\newtheorem{remark}[theorem]{Remark}

% ================================================================
% Macros
% ================================================================
\newcommand{\R}{\mathbb{R}}
\newcommand{\Z}{\mathbb{Z}}
\newcommand{\N}{\mathbb{N}}
\newcommand{\C}{\mathbb{C}}
\newcommand{\F}{\mathbb{F}}
\newcommand{\OO}{\mathbb{O}}
\newcommand{\norm}[1]{\lVert #1 \rVert}
\newcommand{\abs}[1]{\lvert #1 \rvert}
\newcommand{\ip}[2]{\langle #1, #2 \rangle}
\DeclareMathOperator{\tr}{tr}
\DeclareMathOperator{\sgn}{sgn}
\DeclareMathOperator{\Aut}{Aut}

\lstset{
  basicstyle=\ttfamily\small,
  breaklines=true,
  frame=single,
  numbers=left,
  numberstyle=\tiny\color{gray},
  keywordstyle=\color{blue},
  commentstyle=\color{green!50!black},
  stringstyle=\color{red!60!black},
}

% ================================================================
\title{%
  Prime Gap Structure via Exceptional Lattice Projections:\\
  E8, F4, and the Crystalline Grid of $2 \times 10^6$ Primes
}
\author{John Janik}
\date{February 21, 2026}

\begin{document}
\maketitle

\begin{abstract}
We construct a deterministic map from the sequence of prime gaps to the
$E_8$ root system, and study its restriction to the $F_4$ sublattice.
Given consecutive primes $p_n < p_{n+1}$, the normalized gap
$\tilde{g}_n = (p_{n+1} - p_n)/\log p_n$ is mapped to one of the 240
roots of $E_8$ via a phase function derived from the lattice's minimal
norm $\sqrt{2}$.  We project $E_8$ roots onto the $F_4$ sublattice (48
roots) using cosine similarity and show that \emph{all 240} $E_8$ roots
pass the $F_4$ quality threshold---a structural fact we prove by
case analysis on root types.  The Jordan trace from the Albert algebra
$J_3(\OO)$ provides a coloring of the Ulam spiral, and an F4 Exceptional
Fourier Transform (F4-EFT) with Salem--Jordan filtering extracts
\emph{crystalline vertices}: discrete points of maximal $F_4$ resonance.

In addition, we implement twelve independent decoding methods that
attempt to extract structured information from the prime gap sequence
via $E_8/2E_8$ Hamming extraction, sign bits, parity bits, $F_4$ root
indices, Jordan trace classification, and crystalline vertex properties.
All methods are evaluated on $2 \times 10^6$ primes with entropy,
chi-square, and longest-printable-run metrics.

All algorithms are given in full detail with explicit formulas.  The
analysis is implemented in Python, C/OpenMP (self-contained PPM
renderer), and a comprehensive multi-method decoder; source code is
publicly available.
\end{abstract}

\tableofcontents
\newpage

% ================================================================
\section{Introduction}
\label{sec:intro}
% ================================================================

The distribution of prime numbers is one of the oldest problems in
mathematics.  While the Prime Number Theorem establishes that
$\pi(x) \sim x/\log x$ as $x \to \infty$, the fine-scale behavior of
individual prime gaps $g_n = p_{n+1} - p_n$ remains largely mysterious.
The Cram\'er model \cite{cramer1936} predicts that $g_n$ fluctuates
around $\log p_n$, but the actual distribution displays structure that
goes beyond simple randomness: twin primes cluster, Polignac gaps
recur, and the Ulam spiral \cite{ulam1963} reveals diagonal alignments
corresponding to prime-producing quadratic polynomials.

In this paper, we introduce a framework for analyzing prime gap
structure using the exceptional Lie lattices $E_8$ and $F_4$.  Our
approach is organized in four layers:

\begin{enumerate}[label=(\roman*)]
\item \textbf{E8 Root Assignment} (\cref{sec:e8}): Each normalized
  prime gap $\tilde{g}_n$ is deterministically assigned to one of the
  240 roots of $E_8$.  The \emph{projection slope} of the assigned
  root induces a coloring of the Ulam spiral that displays
  concentric ring structure.

\item \textbf{F4 Sublattice Filtering} (\cref{sec:f4}): The E8 roots
  are projected onto the 48-root $F_4$ sublattice via cosine
  similarity, and the \emph{Jordan trace} from the Albert algebra
  $J_3(\OO)$ provides a secondary coloring.  We prove that all 240
  $E_8$ roots achieve $F_4$ projection quality $\ge 0.7$, yielding
  100\% $F_4$ coverage.

\item \textbf{Spectral Analysis} (\cref{sec:spectral}): An
  \emph{F4 Exceptional Fourier Transform} (F4-EFT) decomposes the
  gap signal into 48 spectral components, and a
  \emph{Salem--Jordan kernel} filters the spectrum to extract
  crystalline vertices with idempotent boosting.

\item \textbf{Multi-Method Decoding} (\cref{sec:decoder}): Twelve
  independent extraction methods attempt to recover structured
  information from the prime gap data, evaluated with statistical
  tests for non-randomness.
\end{enumerate}

We emphasize that the patterns we observe are consequences of the
specific mathematical construction applied to prime gap statistics;
they do not constitute a proof of any conjecture about prime
distribution.  The purpose of this paper is to describe the
construction in full reproducible detail, report the computational
results, and discuss what the observed patterns do and do not imply.

\subsection{Notation}

Throughout, $p_n$ denotes the $n$-th prime ($p_1 = 2, p_2 = 3,
\ldots$).  We write $g_n = p_{n+1} - p_n$ for the $n$-th prime gap
and $\tilde{g}_n = g_n / \log p_n$ for the normalized gap.  By the
Prime Number Theorem, $\tilde{g}_n$ has mean asymptotically equal to 1.
We use $\norm{\cdot}$ for the Euclidean norm in $\R^d$.

% ================================================================
\section{The E8 Root System and Prime Gap Assignment}
\label{sec:e8}
% ================================================================

\subsection{The E8 Root System}
\label{sec:e8roots}

The $E_8$ root system is the unique rank-8 even unimodular lattice.
It has 240 roots (vectors of minimal norm $\sqrt{2}$), which fall into
two classes \cite{conway1999,humphreys1972}:

\begin{definition}[E8 roots]
\label{def:e8roots}
The 240 roots of $E_8$ in $\R^8$ consist of:
\begin{enumerate}[label=(\alph*)]
\item \textbf{Type I} (112 roots): $\pm e_i \pm e_j$ for
  $1 \le i < j \le 8$, where $e_i$ is the $i$-th standard basis
  vector.

\item \textbf{Type II} (128 roots):
  $\tfrac{1}{2}(\pm 1, \pm 1, \pm 1, \pm 1, \pm 1, \pm 1, \pm 1,
  \pm 1)$ where the number of negative signs is even.
\end{enumerate}
\end{definition}

\begin{proposition}
All 240 roots have Euclidean norm $\sqrt{2}$.
\end{proposition}

\begin{proof}
For Type~I: $\norm{\pm e_i \pm e_j}^2 = 1 + 1 = 2$.
For Type~II: $\norm{(\pm\tfrac{1}{2})^8}^2 = 8 \cdot \tfrac{1}{4} = 2$.
\end{proof}

\begin{remark}[Enumeration algorithm]
For Type~I, iterate over all $\binom{8}{2} = 28$ pairs $(i,j)$ and all
$4$ sign combinations $(s_1, s_2) \in \{-1,+1\}^2$, yielding
$28 \times 4 = 112$ roots.  For Type~II, iterate over all
$2^8 = 256$ sign masks; for bit $k$ of the mask, set coordinate $k$ to
$+\tfrac{1}{2}$ if the bit is 1, $-\tfrac{1}{2}$ otherwise.  Keep only
masks where the number of 0-bits (negative signs) is even; by parity,
exactly 128 survive.  Total: $112 + 128 = 240$.
\end{remark}

\subsection{Projection Slope}
\label{sec:slope}

We partition the eight coordinates of $\R^8$ into two groups of four:
the ``base'' coordinates $(r_1, r_2, r_3, r_4)$ and the ``fiber''
coordinates $(r_5, r_6, r_7, r_8)$.

\begin{definition}[Projection slope]
\label{def:slope}
For an $E_8$ root $\alpha = (r_1, \ldots, r_8) \in \R^8$, define:
\begin{align}
x(\alpha) &= r_1 + r_2 + r_3 + r_4, \\
y(\alpha) &= r_5 + r_6 + r_7 + r_8.
\end{align}
The \emph{projection slope} is:
\begin{equation}
\label{eq:slope}
s(\alpha) =
\begin{cases}
y(\alpha) / x(\alpha) & \text{if } \abs{x(\alpha)} > 0.01, \\
\sgn(y(\alpha)) \cdot 10 & \text{otherwise}.
\end{cases}
\end{equation}
\end{definition}

\subsection{The Root Assignment Map}
\label{sec:assignment}

Given a normalized prime gap $\tilde{g}_n \ge 0$, we assign an
$E_8$ root index as follows.

\begin{definition}[E8 root assignment]
\label{def:assignment}
Fix an enumeration $\alpha_0, \alpha_1, \ldots, \alpha_{239}$ of the
240 $E_8$ roots (in the order specified by
\cref{def:e8roots}).  Define:
\begin{align}
\label{eq:target_norm}
t(\tilde{g}) &= \sqrt{\max(\tilde{g},\, 0.01)}, \\
\label{eq:phase}
\varphi(\tilde{g}) &= \frac{t(\tilde{g})}{\sqrt{2}} \bmod 1, \\
\label{eq:root_index}
\iota(\tilde{g}) &= \lfloor 240 \cdot \varphi(\tilde{g}) \rfloor \bmod 240.
\end{align}
The assigned root is $\alpha_{\iota(\tilde{g})}$.
\end{definition}

\begin{proposition}
\label{prop:periodicity}
The root assignment is periodic: $\iota(\tilde{g}) = \iota(\tilde{g}')$
whenever $\sqrt{\tilde{g}} - \sqrt{\tilde{g}'} = k\sqrt{2}$ for some
integer $k$.
\end{proposition}

\begin{proof}
If $\sqrt{\tilde{g}} = \sqrt{\tilde{g}'} + k\sqrt{2}$, then
$\varphi(\tilde{g}) = (\sqrt{\tilde{g}}/\sqrt{2}) \bmod 1 =
(\sqrt{\tilde{g}'}/\sqrt{2} + k) \bmod 1 = \varphi(\tilde{g}')$.
\end{proof}

\subsection{Ulam Spiral Coordinates}
\label{sec:ulam}

The Ulam spiral \cite{ulam1963} arranges natural numbers on $\Z^2$ by
spiraling outward from the origin.

\begin{definition}[Ulam coordinates]
\label{def:ulam}
For $p \ge 1$, let $k = \lceil (\sqrt{p} - 1)/2 \rceil$, let
$t = 2k + 1$, and let $m = t^2$.  Set $t \leftarrow t - 1 = 2k$.
Then:
\begin{equation}
(u_x(p),\, u_y(p)) =
\begin{cases}
(k - (m - p),\; -k) & \text{if } p \ge m - t, \\
(-k,\; -k + (m - t - p)) & \text{if } p \ge m - 2t, \\
(-k + (m - 2t - p),\; k) & \text{if } p \ge m - 3t, \\
(k,\; k - (m - 3t - p)) & \text{otherwise}.
\end{cases}
\end{equation}
Each prime is mapped in $O(1)$ time, making the computation
embarrassingly parallel.
\end{definition}

% ================================================================
\section{The F4 Root System and Jordan Trace}
\label{sec:f4}
% ================================================================

\subsection{The F4 Root System}
\label{sec:f4roots}

The $F_4$ root system has rank 4 and 48 roots.  It is the
automorphism group of the exceptional Jordan algebra
$J_3(\OO)$ \cite{freudenthal1954,springer1998}.  The Dynkin diagram
is:
\begin{equation}
\underset{\alpha_1}{\circ}
\text{---}
\underset{\alpha_2}{\circ}
\Longrightarrow
\underset{\alpha_3}{\circ}
\text{---}
\underset{\alpha_4}{\circ}
\end{equation}
The Cartan matrix is:
\begin{equation}
A_{F_4} =
\begin{pmatrix}
 2 & -1 &  0 &  0 \\
-1 &  2 & -2 &  0 \\
 0 & -1 &  2 & -1 \\
 0 &  0 & -1 &  2
\end{pmatrix}.
\end{equation}

\begin{definition}[F4 roots in $\R^4$]
\label{def:f4roots}
The 48 roots of $F_4$ consist of:
\begin{enumerate}[label=(\alph*)]
\item \textbf{24 long roots} (norm $\sqrt{2}$):
  $\pm e_i \pm e_j$ for $1 \le i < j \le 4$.
  ($\binom{4}{2} \times 4 = 24$ roots.)

\item \textbf{8 short roots of type A} (norm $1$):
  $\pm e_i$ for $1 \le i \le 4$.

\item \textbf{8 short roots of type B} (norm $1$):
  $\tfrac{1}{2}(\pm 1, \pm 1, \pm 1, \pm 1)$ with an even number
  of negative signs.

\item \textbf{8 short roots of type C} (norm $1$):
  $\tfrac{1}{2}(\pm 1, \pm 1, \pm 1, \pm 1)$ with an odd number
  of negative signs.
\end{enumerate}
Total: $24 + 8 + 8 + 8 = 48$.
\end{definition}

\subsection{E8 to F4 Projection}
\label{sec:e8f4}

We project $E_8$ roots onto the $F_4$ sublattice via the first four
coordinates.

\begin{definition}[E8 to F4 mapping]
\label{def:e8f4map}
For an $E_8$ root $\alpha = (r_1, \ldots, r_8)$:
\begin{enumerate}
\item Compute the projection $\pi(\alpha) = (r_1, r_2, r_3, r_4)$.
\item If $\norm{\pi(\alpha)} < 0.01$, use the fallback
  $\pi(\alpha) = (r_5, r_6, r_7, r_8)$.
\item Normalize: $\hat{\pi} = \pi(\alpha) / \norm{\pi(\alpha)}$.
\item For each $F_4$ root $\beta_j$ ($j = 0, \ldots, 47$), compute
  the cosine similarity:
  \begin{equation}
  \cos\theta_j = \frac{\abs{\ip{\hat{\pi}}{\hat{\beta}_j}}}
                      {\norm{\hat{\pi}} \cdot \norm{\hat{\beta}_j}}.
  \end{equation}
\item Assign $\alpha$ to the $F_4$ root $\beta_{j^*}$ with the
  largest $\cos\theta_{j^*}$.
\end{enumerate}
\end{definition}

\begin{definition}[F4 quality threshold]
\label{def:quality}
An $E_8$ root $\alpha$ passes the \emph{F4 quality threshold} if
$q(\alpha) \ge 0.7$, where $q(\alpha)$ is the cosine similarity
between the (possibly fallback) projection and the assigned $F_4$ root.
\end{definition}

\begin{theorem}[Complete F4 coverage]
\label{thm:f4complete}
All 240 $E_8$ roots satisfy $q(\alpha) = 1.0$ under the mapping of
\cref{def:e8f4map}.  Consequently, the F4 mapping fraction is
exactly $100\%$.
\end{theorem}

\begin{proof}
We analyze each root type:

\medskip\noindent
\textbf{Case 1: Type~I roots $\pm e_i \pm e_j$ with $i < j \le 4$.}
Both coordinates $i, j$ lie in the first four, so
$\pi(\alpha) = \pm e_i \pm e_j$, which is itself a long $F_4$ root.
The cosine similarity with itself is $1.0$.

\medskip\noindent
\textbf{Case 2: Type~I roots $\pm e_i \pm e_j$ with $i \le 4 < j$.}
Exactly one coordinate is in the first four:
$\pi(\alpha) = \pm e_i$, which is a short $F_4$ root (type~A).
Similarity: $1.0$.

\medskip\noindent
\textbf{Case 3: Type~I roots $\pm e_i \pm e_j$ with $4 < i < j$.}
Both coordinates are in the last four:
$\norm{\pi(\alpha)} = 0$, so the fallback is used.  The fallback
projection $\pi(\alpha) = (r_5, r_6, r_7, r_8) = \pm e_{i-4} \pm e_{j-4}$
maps to coordinates 1--4, which is a long $F_4$ root.  Similarity: $1.0$.

\medskip\noindent
\textbf{Case 4: Type~II roots $(\pm\tfrac{1}{2})^8$ (even negatives).}
The first four coordinates form a vector
$\tfrac{1}{2}(\pm 1, \pm 1, \pm 1, \pm 1)$.  This is an $F_4$ short root
(type~B if even negatives among the first four, type~C if odd negatives
among the first four---both cases match an $F_4$ root exactly).
Similarity: $1.0$.

\medskip
In every case, the projection (or its fallback) is \emph{exactly} an
$F_4$ root, so $q(\alpha) = 1.0 \ge 0.7$.
\end{proof}

\begin{remark}
This contrasts with a na\"ive expectation that $E_8$ would ``lose''
roots when projected to $F_4$.  The completeness follows from the fact
that both the Type~I and Type~II $E_8$ roots, when projected to their
first (or last) four coordinates, always land exactly on an $F_4$ root
vector.  This is a consequence of the branching rule
$E_8 \supset F_4 \times G_2$ and the choice of coordinate-aligned
projection.
\end{remark}

\subsection{The Albert Algebra and Jordan Trace}
\label{sec:jordan}

The exceptional Jordan algebra $J_3(\OO)$ consists of $3 \times 3$
Hermitian matrices over the octonions
$\OO$ \cite{jordan1934,mccrimmon2004,baez2002}:
\begin{equation}
X =
\begin{pmatrix}
\xi_1 & x_3 & \bar{x}_2 \\
\bar{x}_3 & \xi_2 & x_1 \\
x_2 & \bar{x}_1 & \xi_3
\end{pmatrix},
\quad \xi_i \in \R,\; x_i \in \OO.
\end{equation}
The automorphism group is $\Aut(J_3(\OO)) = F_4$.  The
\emph{trace} is $\tr(X) = \xi_1 + \xi_2 + \xi_3$.

\begin{definition}[Jordan trace of an F4 root]
\label{def:jtrace}
For an $F_4$ root $\beta = (b_1, b_2, b_3, b_4) \in \R^4$, define
the \emph{projection matrix}:
\begin{equation}
P =
\begin{pmatrix}
1 & 0 & 0 & 0 \\
0 & 1 & 0 & 0 \\
0 & 0 & 1 & 1
\end{pmatrix},
\end{equation}
which maps $\beta$ to the Albert algebra diagonal
$(\xi_1, \xi_2, \xi_3) = P\beta = (b_1,\; b_2,\; b_3 + b_4)$.
The \emph{Jordan trace} is:
\begin{equation}
\label{eq:jtrace}
J(\beta) = \tr(P\beta) = \xi_1 + \xi_2 + \xi_3 = b_1 + b_2 + b_3 + b_4.
\end{equation}
\end{definition}

\begin{remark}
The projection matrix $P$ embeds the $F_4$ Cartan subalgebra into
the diagonal of $J_3(\OO)$.  The combining of $b_3$ and $b_4$ into
$\xi_3 = b_3 + b_4$ corresponds to the identification of the $F_4$
Cartan subalgebra with the traceless diagonal of $J_3(\OO)$ modulo the
center.  While $J(\beta)$ simplifies to $\sum_i b_i$, the
3-component decomposition $(b_1, b_2, b_3 + b_4)$ retains the
Albert algebra structure and is used for idempotent classification.
\end{remark}

\begin{proposition}[Jordan trace values]
\label{prop:jtrace_values}
The Jordan traces for the 48 $F_4$ roots and their multiplicities are:
\begin{center}
\begin{tabular}{c|cccccccccc}
\toprule
$J(\beta)$ & $-2$ & $-1$ & $0$ & $+1$ & $+2$ \\
\midrule
Long roots & 4 & 0 & 8 & 0 & 4 \\
Short type A & 0 & 4 & 0 & 4 & 0 \\
Short type B & 1 & 0 & 6 & 0 & 1 \\
Short type C & 0 & 4 & 0 & 4 & 0 \\
\midrule
\textbf{Total} & \textbf{5} & \textbf{8} & \textbf{14} & \textbf{8} & \textbf{5} \\
\bottomrule
\end{tabular}
\end{center}
The idempotent-type roots ($\abs{J} = 1$) number 16, and the
nilpotent-type roots ($J = 0$) number 14.
\end{proposition}

\begin{proof}
Direct enumeration.  For long roots $\pm e_i \pm e_j$:
$J = (\pm 1) + (\pm 1) + 0 + 0$ when $i,j \le 2$, etc.
For type~B short roots $\tfrac{1}{2}(\pm 1)^4$ with even negatives:
$J = \tfrac{1}{2}(4), \tfrac{1}{2}(0), \tfrac{1}{2}(-4)$ for 0, 2, 4
negatives respectively, giving $J \in \{+2, 0, -2\}$.
The remaining cases follow similarly.
\end{proof}

\subsection{Idempotent Classification}
\label{sec:idempotent}

The Jordan trace classifies $F_4$ roots by their algebraic type in
$J_3(\OO)$:

\begin{definition}[Root classification]
\label{def:root_class}
An $F_4$ root $\beta$ is classified as:
\begin{itemize}
\item \textbf{Nilpotent}: $\abs{J(\beta)} < 0.01$,
\item \textbf{Primitive}: $0.01 \le \abs{J(\beta)} < 0.5$,
\item \textbf{Idempotent}: $\abs{\abs{J(\beta)} - 1} < 0.1$,
\item \textbf{Regular}: otherwise.
\end{itemize}
\end{definition}

\subsection{F4 Character Weights}
\label{sec:character}

\begin{definition}[F4 character]
\label{def:f4char}
For an $F_4$ root $\beta_j$ with Weyl height
$h(\beta_j) = \sum_{i=1}^4 \abs{(\beta_j)_i}$, define:
\begin{equation}
\label{eq:character}
\chi(\beta_j) =
\begin{cases}
2(1 + 0.1 \cdot h(\beta_j)) & \text{if $\beta_j$ is long}, \\
1 + 0.1 \cdot h(\beta_j) & \text{if $\beta_j$ is short}.
\end{cases}
\end{equation}
\end{definition}

\begin{remark}
The factor of 2 for long roots reflects their larger contribution to the
adjoint representation of $F_4$ ($\dim = 52$).  The Weyl height
modulation $0.1 \cdot h$ provides a mild position-dependent weighting
within the Weyl chamber.  For long roots with $h = 2$ (e.g.,
$e_1 + e_2$), the character is $2(1 + 0.2) = 2.4$.  For short type~A
roots with $h = 1$ (e.g., $e_1$), it is $1.1$.
\end{remark}

\subsection{Plasma Colormap for F4 Grid}
\label{sec:plasma}

The F4 crystalline grid visualization maps the Jordan trace
$J \in [-2, +2]$ to an RGB color via the \texttt{plasma} colormap
\cite{matplotlib2007}:
\begin{equation}
\text{color}(J) = \text{plasma}\!\left(\frac{\text{clamp}(J,\, -2,\, 2) + 2}{4}\right),
\end{equation}
where $\text{plasma}(t)$ for $t \in [0,1]$ interpolates from deep
indigo ($t = 0$, $J = -2$) through magenta to bright yellow
($t = 1$, $J = +2$).  We use a hardcoded 256-entry RGB lookup table
generated from \texttt{matplotlib.cm.plasma}.

% ================================================================
\section{The F4 Exceptional Fourier Transform}
\label{sec:spectral}
% ================================================================

\subsection{Definition}
\label{sec:eft_def}

The F4-EFT decomposes the prime gap signal into 48 spectral
components indexed by $F_4$ roots.

\begin{definition}[F4-EFT]
\label{def:f4eft}
Given $N$ consecutive normalized gaps
$\tilde{g}_0, \ldots, \tilde{g}_{N-1}$ with $E_8$ root assignments
$\iota_0, \ldots, \iota_{N-1}$, the \emph{F4 Exceptional Fourier
Transform} is the 48-component complex vector:
\begin{equation}
\label{eq:f4eft}
\hat{E}_{F_4}(j) = \sum_{\substack{n=0 \\ f(\iota_n) = j}}^{N-1}
(\tilde{g}_n - 1) \cdot \chi(\beta_j)
\cdot \exp\!\left(i \cdot \frac{2\pi \norm{\beta_j}}{\sqrt{2}} \cdot \frac{n}{N}\right),
\quad j = 0, \ldots, 47,
\end{equation}
where $f(\iota_n)$ is the $F_4$ index of $E_8$ root $\iota_n$ (under
the mapping of \cref{def:e8f4map}), and $\chi$ is the $F_4$ character
\eqref{eq:character}.
\end{definition}

\begin{remark}
Key features of the F4-EFT:
\begin{itemize}
\item The factor $(\tilde{g}_n - 1)$ subtracts the expected mean gap
  under the PNT, so the EFT measures deviations from uniformity.
\item The phase $\theta_j(n) = 2\pi \norm{\beta_j} / \sqrt{2} \cdot n/N$
  encodes a time-dependent sweep modulated by the $F_4$ root norm.
  For long roots ($\norm{\beta_j} = \sqrt{2}$), the phase completes
  one full cycle over $N$ gaps.  For short roots ($\norm{\beta_j} = 1$),
  the phase completes $1/\sqrt{2} \approx 0.707$ cycles.
\item Each gap contributes only to the bin of its specific $F_4$ root
  $f(\iota_n)$, not to all 48 components.
\end{itemize}
\end{remark}

\subsection{Power Spectrum and Jordan Decomposition}

The \emph{F4 power spectrum} is:
\begin{equation}
\mathcal{P}(j) = \abs{\hat{E}_{F_4}(j)}^2, \quad j = 0, \ldots, 47.
\end{equation}

The \emph{Jordan decomposition} groups the spectrum into 12 bins by
Jordan trace value:
\begin{equation}
\hat{D}_k = \sum_{j : J(\beta_j) \in B_k} \hat{E}_{F_4}(j),
\quad k = 0, \ldots, 11,
\end{equation}
where $B_0 = [-3, -2.5), B_1 = [-2.5, -2), \ldots, B_{11} = [2.5, 3)$.

\subsection{Phase-Lock Analysis}
\label{sec:phaselock}

The F4-EFT phase structure is characterized by four metrics:

\begin{definition}[Phase-lock diagnostics]
\label{def:phaselock}
\begin{align}
\text{Phase coherence:} \quad C &= \abs{\frac{1}{48} \sum_{j=0}^{47} e^{i \arg \hat{E}_{F_4}(j)}}, \\
\text{Power entropy:} \quad H &= -\sum_{j=0}^{47} p_j \ln p_j \,\Big/\, \ln 48,
  \quad p_j = \mathcal{P}(j) / \textstyle\sum_k \mathcal{P}(k), \\
\text{Long/short ratio:} \quad R &= \frac{\sum_{j \in \mathcal{L}} \mathcal{P}(j)}
  {\sum_{j \in \mathcal{S}} \mathcal{P}(j)}, \\
\text{Jordan peak trace:} \quad J^* &= \arg\max_k \abs{\hat{D}_k}^2.
\end{align}
The system is \emph{phase-locked} if $C > 0.3$ and has
\emph{Jordan structure} if $\max_k \abs{\hat{D}_k}^2 > 2 \cdot \text{mean}_k \abs{\hat{D}_k}^2$.
\end{definition}

\subsection{The Salem--Jordan Kernel}
\label{sec:salem}

\begin{definition}[Salem--Jordan kernel]
\label{def:salem}
For temperature parameter $\tau = 0.5$, the kernel applied to spectral
magnitude $m \ge 0$ is:
\begin{equation}
\label{eq:salem}
K_J(m, \tau) = \frac{52 - 4m^2}{52}
\cdot \frac{1}{e^{m/\tau} + 1}.
\end{equation}
The Fermi--Dirac factor provides exponential suppression for large
magnitudes.  The character factor $(52 - 4m^2)/52$ approximates the
$F_4$ adjoint trace near the identity: $\chi_{F_4}(e^{x/\tau}) \approx
52 - 4\abs{x}^2$.
\end{definition}

\subsection{Crystalline Vertex Extraction}
\label{sec:vertices}

\begin{definition}[Crystalline vertex score]
\label{def:vertex_score}
For gap index $n$ mapping to $F_4$ root $j = f(\iota_n)$:
\begin{equation}
\label{eq:vertex_score}
\sigma(n) =
\begin{cases}
2 \cdot \mathcal{P}(j) &
  \text{if } \abs{\abs{J(\beta_j)} - 1} < 0.2 \quad
  \text{(idempotent boost)}, \\
\mathcal{P}(j) & \text{otherwise}.
\end{cases}
\end{equation}
The top $V$ gap indices by $\sigma(n)$ are the \emph{crystalline
vertices}.
\end{definition}

\begin{remark}
The idempotent boost doubles the score for roots with
$\abs{J} \approx 1$ (i.e., idempotent elements of $J_3(\OO)$).
By \cref{prop:jtrace_values}, 16 of the 48 $F_4$ roots qualify.
Selection uses a min-heap of size $V$ for $O(N \log V)$ extraction.
\end{remark}

% ================================================================
\section{E8 Lattice Decoding}
\label{sec:decoder}
% ================================================================

\subsection{Gap Embedding into $\R^8$}
\label{sec:gap_embed}

Given 8 consecutive primes $p_n, p_{n+1}, \ldots, p_{n+7}$ and the
preceding prime $p_{n-1}$, we compute 8 gaps
$g_i = p_{n+i} - p_{n+i-1}$ for $i = 0, \ldots, 7$.  The embedding is:
\begin{equation}
\label{eq:embedding}
v = \frac{\sqrt{2}}{\sqrt{8} \cdot \sigma_g}
\cdot (g - \bar{g}),
\end{equation}
where $\bar{g} = \frac{1}{8}\sum_{i} g_i$ is the mean and
$\sigma_g = \sqrt{\frac{1}{8}\sum_i (g_i - \bar{g})^2}$ is the
standard deviation.  This normalizes $v$ to have typical norm
$\approx \sqrt{2}$, the $E_8$ root length.

\subsection{Closest Vector Decoding}
\label{sec:cvp}

The $E_8$ lattice is $\Lambda_{E_8} = D_8 \cup (D_8 + c)$, where:
\begin{itemize}
\item $D_8 = \{x \in \Z^8 : \sum x_i \equiv 0 \pmod{2}\}$
  (checkerboard lattice).
\item $c = (\tfrac{1}{2}, \ldots, \tfrac{1}{2})$.
\end{itemize}

\begin{algorithm}[H]
\caption{E8 Closest Vector Decoding}
\label{alg:cvp}
\begin{algorithmic}[1]
\Require Input vector $v \in \R^8$
\Ensure Nearest $E_8$ lattice point $\lambda^*$ and distance $d^*$
\Statex
\State \textbf{Integer sublattice:} $z \gets \text{round}(v)$
\If{$\sum z_i$ is odd}
  \State $k \gets \arg\max_i \abs{v_i - z_i}$
  \Comment{Largest rounding error}
  \State $z_k \gets z_k + \sgn(v_k - z_k)$
  \Comment{Flip parity}
\EndIf
\State $d_1 \gets \norm{v - z}$
\Statex
\State \textbf{Half-integer sublattice:} $w \gets \text{round}(v - c)$
\If{$\sum w_i$ is odd}
  \State $k \gets \arg\max_i \abs{(v_i - \tfrac{1}{2}) - w_i}$
  \State $w_k \gets w_k + \sgn((v_k - \tfrac{1}{2}) - w_k)$
\EndIf
\State $y \gets w + c$; $d_2 \gets \norm{v - y}$
\Statex
\State \Return $(z, d_1)$ if $d_1 \le d_2$, else $(y, d_2)$
\end{algorithmic}
\end{algorithm}

The packing radius of $E_8$ is $R_{\text{pack}} = 1/\sqrt{2} \approx 0.7071$.
Blocks with decoding error $d^* < R_{\text{pack}}$ are \emph{correctable}.

\subsection{Bit Extraction via $E_8/2E_8$}
\label{sec:bits}

The quotient $E_8 / 2E_8 \cong \F_2^4$ is isomorphic to the
extended Hamming code $\mathcal{H}_8$ \cite{conway1999,ebeling2013}.
From the lattice point $\lambda^* = (y_1, \ldots, y_8)$
(translated to integers if half-integer), we extract 4 bits:
\begin{align}
m_1 &= (y_1 + y_2 + y_3 + y_4) \bmod 2, \\
m_2 &= (y_1 + y_2 + y_5 + y_6) \bmod 2, \\
m_3 &= (y_1 + y_3 + y_5 + y_7) \bmod 2, \\
m_4 &= y_1 \bmod 2.
\end{align}
These correspond to the generator matrix of the $[8,4,4]$ extended
Hamming code.  Each block of 8 primes yields 4 bits.  Pairs of
consecutive nibbles $(m_1 m_2 m_3 m_4)$ can be assembled into bytes
in either high--low or low--high order.

% ================================================================
\section{Multi-Method Decoder}
\label{sec:multi_decoder}
% ================================================================

We implement twelve independent methods for extracting structured
information from the prime gap sequence, organized in five groups.
Each method produces a byte stream that is evaluated for non-random
structure.

\subsection{Group 1: E8 Lattice Decoding}

\begin{enumerate}[label=\textbf{M\arabic*}]
\item \textbf{Hamming 4-bit (high--low pairing)}: The standard
  $E_8/2E_8$ extraction of \cref{sec:bits}, with consecutive nibbles
  assembled as $(n_{2k} \ll 4) \mathbin{|} n_{2k+1}$.
  Rate: $\tfrac{1}{2}$ byte per block of 8 primes.

\item \textbf{Hamming 4-bit (low--high pairing)}: Same extraction,
  opposite nibble ordering: $(n_{2k+1} \ll 4) \mathbin{|} n_{2k}$.

\item \textbf{Hamming 4-bit (low-error only)}: Restricts to
  correctable blocks ($d^* < 1/\sqrt{2}$).

\item \textbf{Hamming raw bits}: The 4-bit stream packed into bytes
  (MSB first), without nibble pairing.

\item \textbf{E8 sign bits}: For each block, record
  $\text{bit}_k = [y_k \ge 0]$ for $k = 1, \ldots, 8$.
  Rate: 1 byte per block.

\item \textbf{Sublattice flag}: 1 bit per block indicating integer
  ($0$) or half-integer ($1$) sublattice.
  Rate: $\tfrac{1}{8}$ byte per block.

\item \textbf{E8 parity bits}: $\text{bit}_k = y_k \bmod 2$ for each
  coordinate.  Rate: 1 byte per block.
\end{enumerate}

\subsection{Group 2: E8 Root Assignment}

\begin{enumerate}[resume,label=\textbf{M\arabic*}]
\item \textbf{E8 root index mod 256}: For each gap $n$, output
  $\iota(\tilde{g}_n) \bmod 256$ as a byte.
  Rate: 1 byte per gap.

\item \textbf{E8 projection slope (quantized)}: Map the projection
  slope $s \in [-10, 10]$ to $[0, 255]$ via
  $\lfloor 255 \cdot (s + 10) / 20 \rfloor$.
\end{enumerate}

\subsection{Group 3: Raw Gap Properties}

\begin{enumerate}[resume,label=\textbf{M\arabic*}]
\item \textbf{Gap value mod 256}: Direct gap size $g_n \bmod 256$.

\item \textbf{Gap/2 mod 256}: $(g_n / 2) \bmod 256$ (gaps are even
  for $p > 2$).

\item \textbf{Normalized gap (quantized)}: Map
  $\tilde{g}_n \in [0, 4]$ to $[0, 255]$.
\end{enumerate}

\subsection{Group 4: F4 Sub-harmonic}

\begin{enumerate}[resume,label=\textbf{M\arabic*}]
\item \textbf{F4 root index}: For F4-mapped gaps, the F4 root index
  $(0\text{--}47)$ as a byte.

\item \textbf{F4 root index mod 26}: Map to letters A--Z via
  $65 + (j \bmod 26)$.

\item \textbf{Jordan trace classification bits}: 2 bits per F4 gap:
  bit~0 = $[J > 0]$, bit~1 = $[\abs{J} > 1]$.
\end{enumerate}

\subsection{Group 5: Crystalline Vertices}

\begin{enumerate}[resume,label=\textbf{M\arabic*}]
\item \textbf{Vertex gap values}: Gap sizes $g_n$ at the $V$
  crystalline vertex positions, mod 256.

\item \textbf{Vertex gaps mod 26}: Same, mapped to A--Z.

\item \textbf{Vertex spacing}: Differences between consecutive
  sorted vertex positions, mod 256.

\item \textbf{Vertex prime values mod 256}: The prime $p_{n+1}$ at
  each vertex position, mod 256.
\end{enumerate}

\subsection{Statistical Evaluation}
\label{sec:stats}

Each method's output byte stream is evaluated by:

\begin{definition}[Decoder metrics]
\label{def:metrics}
\begin{itemize}
\item \textbf{Shannon entropy}: $H = -\sum_{b=0}^{255} p_b \log_2 p_b$
  bits per byte (max 8.0 for uniform).
\item \textbf{Chi-square $p$-value}: Test byte frequencies against
  uniform $U(0, 255)$ with 255 degrees of freedom, using normal
  approximation $z = (\chi^2 - 255)/\sqrt{510}$.
\item \textbf{Longest printable run}: Maximum contiguous substring
  of bytes in the printable ASCII range $[32, 126]$.
\end{itemize}
\end{definition}

A stream showing non-random structure would exhibit: low entropy
($H \ll 8$), small $p$-value ($p \ll 0.05$), and/or long printable
runs ($\gg 10$ characters forming readable text).

% ================================================================
\section{Computational Implementation}
\label{sec:implementation}
% ================================================================

\subsection{Prime Data}

Primes are loaded from external files \texttt{primes1.txt} through
\texttt{primes50.txt} (generated by
\texttt{primesieve}\footnote{\url{https://primesieve.org/}}).  Each
file contains whitespace-separated integers with a header line.  The
loader skips the header (detected by the presence of alphabetic
characters), parses all integers $> 1$, and stops at the requested
maximum.

\subsection{Self-Contained C Implementation}
\label{sec:c_impl}

The F4 crystalline grid is implemented as a single C source file
(\texttt{f4\_crystalline\_grid.c}, $\sim$600 lines) that produces
a PPM image with no external dependencies beyond \texttt{libm} and
OpenMP.  The full pipeline:

\begin{algorithm}[H]
\caption{F4 Crystalline Grid (C/OpenMP)}
\label{alg:pipeline}
\begin{algorithmic}[1]
\Require \texttt{--max-primes} $N$, \texttt{--n-vertices} $V$,
  \texttt{--size} $S$
\Ensure PPM image file
\State Generate 240 $E_8$ roots, 48 $F_4$ roots
  \Comment{$O(1)$}
\State Build $E_8 \to F_4$ mapping (cosine similarity)
  \Comment{$O(240 \times 48)$}
\State Load primes from files
  \Comment{$O(N)$}
\For{each gap $n = 0, \ldots, N-2$ \textbf{(OpenMP parallel)}}
  \State $g_n \gets p_{n+1} - p_n$;
    $\tilde{g}_n \gets g_n / \max(\log p_n, 1)$
  \State $\iota_n \gets \lfloor 240 \cdot (\sqrt{\max(\tilde{g}_n, 0.01)}/\sqrt{2} \bmod 1) \rfloor$
  \State $j_n \gets f(\iota_n)$ (F4 index);
    $J_n \gets J(\beta_{j_n})$ (Jordan trace)
\EndFor
\State Compute F4-EFT spectrum (48 complex components)
  \Comment{Serial, $O(N)$}
\State Compute vertex scores $\sigma(n)$ with idempotent boost
  \Comment{OpenMP}
\State Select top $V$ by min-heap
  \Comment{$O(N \log V)$}
\State Compute Ulam coordinates $u(p_n)$ for all $n$
  \Comment{OpenMP}
\State Allocate $S \times S \times 3$ pixel buffer (black)
\State Plot F4 primes: map $J_n \in [-2,2]$ to plasma LUT
  \Comment{OpenMP}
\State Draw white circles with yellow edges at vertex positions
\State Write P6 PPM
\end{algorithmic}
\end{algorithm}

\begin{remark}[Memory estimate]
For $N = 2 \times 10^6$ primes:
primes (16~MB), gaps + normalized gaps (32~MB), E8/F4 assignments
(8~MB), scores (16~MB), Ulam coordinates (16~MB), pixel buffer at
$6000^2 \times 3$ (108~MB).  Total: $\approx 200$~MB.
For $N = 5 \times 10^7$: $\approx 2$~GB.  For $N = 10^8$:
$\approx 4$~GB.
\end{remark}

\subsection{Python F4 Analysis Pipeline}

The Python implementation (\texttt{e8\_f4\_prime\_analysis.py})
provides the reference pipeline using NumPy, with supporting modules:

\begin{itemize}
\item \texttt{f4\_lattice.py}: $F_4$ root generation, $E_8 \to F_4$
  mapping, Cartan matrix, character precomputation.
\item \texttt{jordan\_algebra.py}: Octonion multiplication table,
  Albert algebra $J_3(\OO)$, Jordan trace, Cayley plane projection.
\item \texttt{f4\_eft.py}: F4-EFT computation, phase-lock analysis,
  crystalline vertex extraction.
\item \texttt{salem\_jordan.py}: Salem--Jordan kernel, Fermi--Dirac
  filter, adaptive $\tau$ selection, spectral filtering.
\end{itemize}

\subsection{Multi-Method Decoder Implementation}

The decoder (\texttt{e8\_multi\_decoder.py}) imports from both the
F4 analysis pipeline and the E8 lattice decoder
(\texttt{e8\_prime\_decoder.py}).  It runs all 19 base methods plus
first-$N$ filtered variants (first 100 and first 1000 bytes of each),
totaling 53+ method variants evaluated in a single pass.

\subsection{Build System}

\begin{lstlisting}[language=make,caption={Makefile targets}]
CC = gcc
CFLAGS = -O3 -march=native -Wall -fopenmp

f4_crystalline_grid: f4_crystalline_grid.c
	$(CC) $(CFLAGS) -o $@ $< -lm

run-crystal:
	./f4_crystalline_grid --max-primes 2000000 --n-vertices 38

run-decoder:
	python3 e8_multi_decoder.py --max-primes 2000000
\end{lstlisting}

The PPM output is converted to PNG via ImageMagick:
\texttt{convert grid.ppm grid.png}.

% ================================================================
\section{Results}
\label{sec:results}
% ================================================================

\subsection{F4 Crystalline Grid ($N = 2{,}000{,}000$ primes)}

Running \texttt{f4\_crystalline\_grid --max-primes 2000000 --n-vertices 38}
produces the following:

\begin{table}[h]
\centering
\caption{F4 Crystalline Grid: summary statistics ($N = 2 \times 10^6$)}
\label{tab:f4grid}
\begin{tabular}{lr}
\toprule
\textbf{Quantity} & \textbf{Value} \\
\midrule
Primes loaded & 2{,}000{,}000 \\
Prime range & 2 to 32{,}452{,}843 \\
Total gaps & 1{,}999{,}999 \\
$E_8$ roots & 240 \\
$F_4$ roots & 48 \\
$E_8$ roots passing $F_4$ threshold & 240 (100\%) \\
Gaps mapping to $F_4$ & 1{,}999{,}999 (100.0\%) \\
Crystalline vertices selected & 38 \\
Ulam coordinate range & $x \in [-2848, 2848]$, $y \in [-2848, 2848]$ \\
Image size & $6000 \times 6000$ pixels \\
PPM file size & 103 MB \\
PNG file size (after conversion) & 2.4 MB \\
\bottomrule
\end{tabular}
\end{table}

The 100\% F4 mapping fraction confirms \cref{thm:f4complete}: every
$E_8$ root projects to an $F_4$ root with cosine similarity $1.0$.

\subsection{Multi-Method Decoder Results ($N = 100{,}000$ primes)}

The decoder evaluated 53 method variants.  \cref{tab:decoder_top}
shows the methods ranked by longest printable ASCII run,
and \cref{tab:decoder_entropy} ranks by lowest entropy.

\begin{table}[h]
\centering
\caption{Top methods by longest printable ASCII run}
\label{tab:decoder_top}
\begin{tabular}{rlrrr}
\toprule
\textbf{\#} & \textbf{Method} & \textbf{Bytes} & \textbf{MaxRun} & \textbf{Entropy} \\
\midrule
14 & F4 root index mod 26 (letter) & 85{,}502 & 85{,}502 & 3.541 \\
13 & F4 root index (F4 gaps only) & 85{,}502 & 46 & 3.693 \\
12 & Normalized gap (quantized) & 99{,}999 & 20 & 6.701 \\
 9 & E8 projection slope (quantized) & 99{,}999 & 11 & 2.131 \\
 8 & E8 root index mod 256 & 99{,}999 & 11 & 7.266 \\
 3 & Hamming 4-bit (low-error) & 1{,}335 & 9 & 7.845 \\
 2 & Hamming 4-bit (low--high) & 6{,}249 & 9 & 7.950 \\
 1 & Hamming 4-bit (high--low) & 6{,}249 & 8 & 7.950 \\
15 & Jordan trace class.\ bits & 21{,}375 & 8 & 3.911 \\
\bottomrule
\end{tabular}
\end{table}

\begin{table}[h]
\centering
\caption{Top methods by lowest entropy (most structure)}
\label{tab:decoder_entropy}
\begin{tabular}{rlrrr}
\toprule
\textbf{\#} & \textbf{Method} & \textbf{Bytes} & \textbf{Entropy} & \textbf{$\chi^2$ $p$} \\
\midrule
17 & Vertex gaps mod 26 (letters) & 500 & 0.118 & 0.000 \\
16 & Vertex gap values mod 256 & 500 & 0.128 & 0.000 \\
 9 & E8 projection slope & 99{,}999 & 2.131 & 0.000 \\
 7 & E8 parity bits & 12{,}499 & 3.196 & 0.000 \\
14 & F4 root index mod 26 & 85{,}502 & 3.541 & 0.000 \\
13 & F4 root index (F4 only) & 85{,}502 & 3.693 & 0.000 \\
15 & Jordan trace class.\ bits & 21{,}375 & 3.911 & 0.000 \\
10 & Gap value mod 256 & 99{,}999 & 3.943 & 0.000 \\
\bottomrule
\end{tabular}
\end{table}

\subsection{Analysis of Decoder Results}

\paragraph{Hamming methods (M1--M4).}
Entropy $\approx 7.95$ bits/byte, indistinguishable from random.
The chi-square $p$-value is $< 0.01$, indicating a slight deviation
from uniformity in the nibble distribution, but the entropy is so
close to the maximum of 8.0 that no meaningful structure is present.
The longest printable runs (8--9 characters) are consistent with
random expectation for streams of $\sim$6{,}000 bytes.

\paragraph{Projection slope (M9).}
Entropy $2.13$ bits/byte reflects the fact that the slope takes only a
small number of distinct values ($\{-10, -2, -1, 0, 1, 2, 10\}$
for the 240 roots), so the quantized byte values cluster around
a few values (notably \texttt{0x72}, \texttt{0x7F}, \texttt{0x8C},
\texttt{0xFF}, \texttt{0x00}).  This is a consequence of the E8 root
geometry, not of prime structure.

\paragraph{F4 root index (M13--M14).}
Entropy $3.5$--$3.7$ bits/byte.  Since there are only 48 $F_4$ root
indices, the maximum possible entropy is $\log_2 48 \approx 5.6$
bits; the observed value indicates non-uniform distribution across
the 48 roots.  When mapped to letters A--Z (mod 26), the entire
85{,}502-byte stream is ``printable'' by construction, but the
resulting text is not readable English.

\paragraph{Vertex gap values (M16--M17).}
Entropy $0.12$--$0.13$ bits/byte---the lowest of all methods.
This is because crystalline vertices are selected for maximal F4
power spectrum, and the corresponding gaps cluster around a few
values.  With only 500 bytes, the low entropy reflects the vertex
selection criterion rather than an encoded message.

\paragraph{Parity bits (M7).}
Entropy $3.20$ bits/byte.  The low entropy arises because most $E_8$
lattice point coordinates are small integers ($0$ or $\pm 1$), so
parity is highly non-uniform.

\subsection{Performance}

\begin{table}[h]
\centering
\caption{Runtime for $N = 2 \times 10^6$ primes on Intel Core Ultra 9 275HX}
\label{tab:perf}
\begin{tabular}{lcc}
\toprule
\textbf{Program} & \textbf{Time} & \textbf{Output} \\
\midrule
\texttt{f4\_crystalline\_grid} (C/OpenMP) & $< 5$ s & 103 MB PPM \\
ImageMagick \texttt{convert} PPM $\to$ PNG & $\sim 2$ s & 2.4 MB PNG \\
\texttt{e8\_multi\_decoder.py} ($10^5$ primes) & 3.6 s & 53 methods \\
\bottomrule
\end{tabular}
\end{table}

% ================================================================
\section{Octonion Arithmetic}
\label{sec:octonions}
% ================================================================

The Jordan algebra module implements full octonion arithmetic
as the algebraic foundation for the $F_4$-Albert algebra connection.

\subsection{Multiplication Table}

The octonions $\OO$ are an 8-dimensional non-associative algebra
with basis $\{1, e_1, \ldots, e_7\}$ and multiplication:
$e_i \cdot e_j = f_{ijk} \, e_k$, where $f_{ijk}$ are the structure
constants given by the Cayley--Dickson construction.  The multiplication
table is:
\begin{equation}
\text{MULT\_TABLE}[i][j] = k, \qquad
\text{MULT\_SIGNS}[i][j] = \pm 1,
\end{equation}
such that $e_i \cdot e_j = \text{MULT\_SIGNS}[i][j] \cdot e_{\text{MULT\_TABLE}[i][j]}$.

The full $8 \times 8$ sign matrix is:
\begin{equation}
\text{SIGNS} =
\begin{pmatrix}
+ & + & + & + & + & + & + & + \\
+ & - & + & - & + & - & - & + \\
+ & - & - & + & + & + & - & - \\
+ & + & - & - & + & - & + & - \\
+ & - & - & - & - & + & + & + \\
+ & + & - & + & - & - & + & - \\
+ & + & + & - & - & - & - & + \\
+ & - & + & + & - & + & - & -
\end{pmatrix}
\end{equation}

Operations implemented:
\begin{itemize}
\item Addition, subtraction (componentwise).
\item Multiplication: $\OO \times \OO \to \OO$ via
  $(a \cdot b)_k = \sum_{i,j} f_{ijk} \, a_i \, b_j$
  ($O(64)$ multiplications).
\item Conjugation: $\bar{a} = a_0 - \sum_{k=1}^7 a_k e_k$.
\item Norm: $\norm{a} = \sqrt{a_0^2 + \cdots + a_7^2}$.
\end{itemize}

\subsection{Albert Algebra Product}

The Albert algebra $J_3(\OO)$ supports the Jordan product
$X \circ Y = (XY + YX)/2$, computed as:
\begin{equation}
(X \circ Y)_{ii} = X_{ii} Y_{ii} + \text{Re}(X_{jk} \overline{Y_{jk}})
  + \text{Re}(X_{ki} \overline{Y_{ki}}),
\end{equation}
where $(i,j,k)$ is a cyclic permutation of $(1,2,3)$.

\subsection{Cayley Plane Projection}

The Cayley plane $\OO P^2$ is a 16-dimensional manifold on which
$F_4$ acts.  For visualization, we project $F_4$ roots to polar
coordinates $(r, \theta)$ via:
\begin{align}
z &= b_1 + i \, b_2, \quad w = b_3 + i \, b_4, \\
r &= \sqrt{\abs{z}^2 + \abs{w}^2}, \quad
\theta = \arctan\!\left(\frac{\text{Im}(z) + \text{Im}(w)}
  {\text{Re}(z) + \text{Re}(w)}\right).
\end{align}

% ================================================================
\section{Discussion}
\label{sec:discussion}
% ================================================================

\subsection{What the Patterns Are}

The ring structure in the E8 slope visualization arises from the
interplay of two effects:

\begin{enumerate}
\item \textbf{Gap periodicity in root space}: The map
  $\tilde{g} \mapsto \varphi(\tilde{g})$ wraps the normalized gap
  around the circle $[0, 1)$ via a square-root scaling.  As one moves
  outward in the Ulam spiral, the typical gap size grows
  (logarithmically), causing the phase to advance and the assigned
  slope to cycle.

\item \textbf{Ulam spiral geometry}: Primes at similar distances from
  the center tend to have similar magnitudes and hence similar
  normalized gaps, producing coherent coloring in annular regions.
\end{enumerate}

The 100\% F4 mapping (\cref{thm:f4complete}) means that the F4
crystalline grid visualizes \emph{all} prime gaps, not a filtered
subset.  The crystalline vertices select gaps where the F4-EFT power
spectrum is concentrated, boosted by the idempotent condition on the
Jordan trace.

\subsection{What the Patterns Are Not}

\begin{itemize}
\item The E8 root assignment is a \emph{deterministic function of the
  gap size alone}.  It encodes no deeper number-theoretic structure
  beyond the normalized gap distribution.

\item The ring structure would appear for \emph{any} sequence of
  positive reals with similar statistical properties to prime gaps
  (e.g., a Poisson process with rate $1/\log n$).

\item The crystalline vertices are not ``special primes'' in any
  number-theoretic sense.

\item The multi-method decoder (\cref{sec:multi_decoder}) finds no
  method producing readable text.  All methods either show
  near-maximum entropy (random) or low entropy explained by the
  algebraic structure of the mapping (small number of root types),
  not by an encoded message.
\end{itemize}

\subsection{The Complete F4 Coverage Phenomenon}

The result of \cref{thm:f4complete}---that all 240 $E_8$ roots
project to $F_4$ with similarity $1.0$---deserves comment.  This is
\emph{not} a generic feature of projections from $E_8$ to rank-4
sublattices.  It holds specifically because:
\begin{enumerate}
\item The coordinate-aligned projection $(r_1, \ldots, r_8) \mapsto
  (r_1, \ldots, r_4)$ sends each $E_8$ root type to an exact $F_4$
  root type (long to long, short to short).
\item The fallback to coordinates $(r_5, \ldots, r_8)$ catches the
  roots concentrated entirely in the fiber, which again project
  exactly to $F_4$ roots.
\end{enumerate}
A random $4$-dimensional projection of $E_8$ would typically yield
far fewer exact $F_4$ matches.

\subsection{Reproducibility}

All source code is available at:
\begin{center}
\url{https://github.com/johnjanik/HodgedeRham}
\end{center}
The repository contains:
\begin{itemize}
\item \texttt{f4\_crystalline\_grid.c}: Self-contained C/OpenMP
  renderer (PPM output, $\sim$600 lines, no dependencies beyond
  \texttt{libm}).
\item \texttt{e8\_multi\_decoder.py}: Multi-method decoder (12 base
  methods $\times$ filtered variants = 53 total).
\item \texttt{e8\_f4\_prime\_analysis.py}: Python F4 analysis pipeline.
\item \texttt{f4\_lattice.py}, \texttt{jordan\_algebra.py},
  \texttt{f4\_eft.py}, \texttt{salem\_jordan.py}: Supporting modules.
\item \texttt{e8\_prime\_decoder.py}: E8 lattice decoder with
  statistical tests.
\item \texttt{Makefile}: Build and run targets.
\end{itemize}
Build with \texttt{gcc -O3 -march=native -fopenmp -lm}.

% ================================================================
\section{Conclusion}
\label{sec:conclusion}
% ================================================================

We have presented a complete, reproducible pipeline for analyzing prime
gap structure through the lens of exceptional Lie lattices.  The E8
root assignment map transforms the scalar sequence of normalized prime
gaps into a structured coloring of the Ulam spiral, and the F4
restriction, Jordan trace, and Salem--Jordan filter extract a discrete
``crystalline grid'' from the continuous ring pattern.

The key structural result (\cref{thm:f4complete}) is that all 240
$E_8$ roots project exactly to $F_4$ roots under the natural
coordinate-aligned embedding, yielding 100\% F4 coverage of all
prime gaps.

The multi-method decoder finds no evidence of an encoded message in
the prime gap data: all 53 extraction methods either produce
near-random output (entropy $\approx 8$ bits/byte) or low-entropy
output fully explained by the finite cardinality of root types.

The mathematical objects involved---$E_8$, $F_4$, the Albert algebra
$J_3(\OO)$, the Salem kernel---are among the most beautiful structures
in mathematics.  Whether the observed patterns reflect deep connections
to prime distribution or are primarily a consequence of projecting gap
statistics onto a rich algebraic framework remains an open question.

% ================================================================
% Appendices
% ================================================================

\appendix

\section{Complete E8 Root Enumeration}
\label{app:roots}

For reproducibility, we specify the exact enumeration order of the
240 $E_8$ roots used in all computations.

\textbf{Type~I} (roots 0--111): Iterate $i = 0, \ldots, 7$ and
$j = i+1, \ldots, 7$.  For each pair $(i, j)$, enumerate the four sign
patterns $(s_1, s_2) \in \{(-1,-1), (-1,+1), (+1,-1), (+1,+1)\}$.
The root is $s_1 e_i + s_2 e_j$.

\textbf{Type~II} (roots 112--239): Iterate $\text{mask} = 0, \ldots,
255$.  For each mask, let
$r_k = \tfrac{1}{2}$ if bit $k$ is set, $-\tfrac{1}{2}$ otherwise
($k = 0, \ldots, 7$).  Keep only masks where the number of unset bits
(negative signs) is even.  This yields 128 roots in mask order.

\section{Complete F4 Root Enumeration}
\label{app:f4roots}

\textbf{Long roots} (indices 0--23): Iterate $i = 0, \ldots, 3$ and
$j = i+1, \ldots, 3$.  For each pair, enumerate four sign patterns.

\textbf{Short type~A} (indices 24--31): Iterate $i = 0, \ldots, 3$,
signs $s \in \{-1, +1\}$.

\textbf{Short type~B} (indices 32--39): Iterate mask $= 0, \ldots, 15$,
keep even negative count.

\textbf{Short type~C} (indices 40--47): Iterate mask $= 0, \ldots, 15$,
keep odd negative count.

\section{Plasma Colormap Table}
\label{app:colormap}

The 256-entry plasma LUT is generated from \texttt{matplotlib 3.8}:
\begin{lstlisting}[language=Python]
import matplotlib.cm as cm
for i in range(256):
    r, g, b, _ = cm.plasma(i / 255.0)
    print(f"  {{{int(r*255)}, {int(g*255)}, {int(b*255)}}},")
\end{lstlisting}
The full table is embedded in \texttt{f4\_crystalline\_grid.c} and maps
$t \in [0, 1]$ (derived from $J \in [-2, 2]$ via $t = (J+2)/4$)
to an RGB triple.  The endpoints: $t=0$ maps to deep indigo (12, 7, 134);
$t=1$ maps to bright yellow (239, 248, 33).

\section{Decoder Method Cross-Reference}
\label{app:decoder_xref}

\begin{longtable}{cllc}
\toprule
\textbf{M\#} & \textbf{Method} & \textbf{Source function} & \textbf{Bits/gap} \\
\midrule
\endhead
1 & Hamming high--low & \texttt{extract\_bits} + pair & 0.5 \\
2 & Hamming low--high & \texttt{extract\_bits} + pair & 0.5 \\
3 & Hamming low-error & \texttt{extract\_bits} + filter & 0.5 \\
4 & Hamming raw bits & \texttt{extract\_bits} & 0.5 \\
5 & Sign bits & \texttt{e8\_decode} signs & 1.0 \\
6 & Sublattice flag & \texttt{e8\_decode} type & 0.125 \\
7 & Parity bits & \texttt{e8\_decode} parities & 1.0 \\
8 & Root index mod 256 & \texttt{assign\_root} & 1.0 \\
9 & Projection slope & \texttt{projected\_slopes} & 1.0 \\
10 & Gap mod 256 & raw gap & 1.0 \\
11 & Gap/2 mod 256 & raw gap $/2$ & 1.0 \\
12 & Normalized gap & $\tilde{g}$ quantized & 1.0 \\
13 & F4 root index & \texttt{project\_e8\_to\_f4} & 1.0 \\
14 & F4 mod 26 (letter) & F4 index $\bmod 26$ & 1.0 \\
15 & Jordan trace bits & $J > 0$, $|J| > 1$ & 0.25 \\
16 & Vertex gaps & crystalline selection & var. \\
17 & Vertex mod 26 & crystalline + mod 26 & var. \\
18 & Vertex spacing & $\Delta$(vertex positions) & var. \\
19 & Vertex primes & $p$ at vertices & var. \\
\bottomrule
\end{longtable}

% ================================================================
% Bibliography
% ================================================================

\begin{thebibliography}{99}

\bibitem{adams1996}
J.~F.~Adams.
\newblock \emph{Lectures on Exceptional Lie Groups}.
\newblock University of Chicago Press, 1996.

\bibitem{baez2002}
J.~C.~Baez.
\newblock The octonions.
\newblock \emph{Bull.\ Amer.\ Math.\ Soc.}\ (N.S.), 39(2):145--205, 2002.

\bibitem{conway1999}
J.~H.~Conway and N.~J.~A.~Sloane.
\newblock \emph{Sphere Packings, Lattices and Groups}.
\newblock Springer, 3rd edition, 1999.

\bibitem{cramer1936}
H.~Cram\'er.
\newblock On the order of magnitude of the difference between
  consecutive prime numbers.
\newblock \emph{Acta Arith.}, 2(1):23--46, 1936.

\bibitem{ebeling2013}
W.~Ebeling.
\newblock \emph{Lattices and Codes}.
\newblock Springer, 3rd edition, 2013.

\bibitem{freudenthal1954}
H.~Freudenthal.
\newblock Beziehungen der $E_7$ und $E_8$ zur Oktavenebene, I--XI.
\newblock \emph{Indag.\ Math.}, 16--25, 1954--1963.

\bibitem{goldston2009}
D.~A.~Goldston, J.~Pintz, and C.~Y.~Y{\i}ld{\i}r{\i}m.
\newblock Primes in tuples I.
\newblock \emph{Ann.\ Math.}, 170(2):819--862, 2009.

\bibitem{humphreys1972}
J.~E.~Humphreys.
\newblock \emph{Introduction to Lie Algebras and Representation Theory}.
\newblock Springer, 1972.

\bibitem{jordan1934}
P.~Jordan, J.~von Neumann, and E.~Wigner.
\newblock On an algebraic generalization of the quantum mechanical
  formalism.
\newblock \emph{Ann.\ Math.}, 35(1):29--64, 1934.

\bibitem{matplotlib2007}
J.~D.~Hunter.
\newblock Matplotlib: A 2D graphics environment.
\newblock \emph{Comput.\ Sci.\ Eng.}, 9(3):90--95, 2007.

\bibitem{maynard2015}
J.~Maynard.
\newblock Small gaps between primes.
\newblock \emph{Ann.\ Math.}, 181(1):383--413, 2015.

\bibitem{mccrimmon2004}
K.~McCrimmon.
\newblock \emph{A Taste of Jordan Algebras}.
\newblock Springer, 2004.

\bibitem{openmp2018}
{OpenMP Architecture Review Board}.
\newblock \emph{OpenMP Application Programming Interface, Version 5.0},
  2018.

\bibitem{salem1950}
R.~Salem.
\newblock Power series with integral coefficients.
\newblock \emph{Duke Math.\ J.}, 12(2):153--172, 1945.

\bibitem{springer1998}
T.~A.~Springer.
\newblock Jordan algebras and algebraic groups.
\newblock \emph{Ergebnisse der Mathematik}, vol.~75. Springer, 1998.

\bibitem{ulam1963}
S.~M.~Ulam.
\newblock A visual display of some properties of the distribution of
  primes.
\newblock \emph{Notices Amer.\ Math.\ Soc.}, 10:197, 1963.

\bibitem{viazovska2017}
M.~S.~Viazovska.
\newblock The sphere packing problem in dimension~8.
\newblock \emph{Ann.\ Math.}, 185(3):991--1015, 2017.

\bibitem{zhang2014}
Y.~Zhang.
\newblock Bounded gaps between primes.
\newblock \emph{Ann.\ Math.}, 179(3):1121--1174, 2014.

\end{thebibliography}

\end{document}
